\begin{abstract}
Austenitic stainless steels have been used since the early days of nuclear power and they will play an important role in the construction of \acrfull{gen3+} and \acrfull{gen4} plant designs.  Materials used in future reactor designs will need to withstand more extreme conditions whilst improving upon safety.  Neutron damage is a major contributor to material failure for many complex reasons.  

It is known that radiation damage depletes Chromium at the grain boundary.  As Chromium is a key element in giving stainless steel its corrosion resistance, a depletion of it leads to \acrfull{igscc}.  Corrosion may also be prevented by adding small amounts of Palladium and Ruthenium to the steel, but this may or may not be affected by neutron damage.

Irradiation of components using neutrons is expensive with relatively few high flux reactors available, but even with these reactors it is difficult to speed up the damage process appreciably.  A proton beam may be used to emulate neutron damage at a much faster rate and it is also possible to model radiation damage using a computer.  

Radioactive waste is created in a reactor when neutrons activate isotopes that make up the components within.  This is also true where a light ion beam is used in place of neutrons.  The relationship between the ion beam parameters and the component being irradiated is complex and depends upon the type and energy of the projectile, the thickness of the target and its constituent isotopes.

A modified Bateman equation was derived to calculate the radioactivity of a proton irradiated target and a computer program (Activity) was created to implement the calculation using evaluated nuclear reaction cross section data.  The program has been compared to data measured from the irradiation of an iron sample using the University of Birmingham cyclotron.  It was also used to model the radioactivity of a target irradiated to 100 \acrfull{dpa} at a range of beam energies in order to minimise both the irradiation time and radioactive waste.  The results show an increase in target activity of 3 orders of magnitude for a 25MeV proton beam when compared to a 10MeV proton beam.

Interatomic potentials are required to model radiation damage and its effect using \acrfull{md} or \acrfull{akmc}.  Platinum group metals \acrshort{pgm}s may be added to a stainless steel to improve corrosion resistance, importantly in conditions where chromium is depleted.  There are existing potentials for binary alloys of Iron and Palladium as well as for the pure elements Iron, Palladium and Ruthenium, but they are designed for use with particular crystal structures such as \acrfull{bcc} and \acrfull{hcp}.

In this work \acrfull{dft} is used to create a reference database that is needed for fitting Fe-Pd and Fe-Ru potentials.  As the material of interest is austenitic stainless steel, the reference database and potential are restricted to the \acrfull{fcc} crystal structure.  The potential type is a many body \acrfull{eam} that also has a \acrfull{zbl} core potential in order to model collisions.

A computer program (EAMPA) was developed in Python and Fortran to fit potentials using the reference database and bulk properties.  The resulting potentials are a step towards \acrshort{md} and \acrshort{akmc} simulations of radiation damage in \acrshort{pgm} doped austenitic stainless steels.  They are also a step towards investigating whether or not radiation damage causes \acrshort{pgm}s to deplete or enrich at grain boundaries, showing whether or not their corrosion resistance is removed or retained.  
\end{abstract}