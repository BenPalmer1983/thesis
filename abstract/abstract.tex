\begin{abstract}
Austenitic stainless steels have been used since the early days of nuclear power and they will play an important role in the construction of \acrfull{gen3+} and \acrfull{gen4} plant designs.  Irradiation of components by neutrons is expensive with relatively few high flux reactors available.  The experiments to represent 30-40 years within a reactor core are time consuming and produce radioactive waste as a by-product.  Two alternatives to testing by irradiating with a neutron source are either irradiating with a proton beam or by modelling damage events with a computer.  



The efficiency and computational power of computers continues to improve.  First principles \acrfull{dft} and \acrfull{md} are computational methods that are now able to give insights into why materials behave the way they do.  

1. Proton sources are more widespread, they are relatively cheap and may be focused into a much smaller beam to concentrate the amount of damage per time unit.  The radioactive waste may be significant but by carefully selecting the ion energy the electromagnetic repulsion between ions and nuclei reduces transmutation of target atoms whilst maintaining the amount and depth of damage into the material required for testing purposes.  

A modified Bateman equation was derived to calculate the radioactivity of a proton irradiated target and a computer program (Activity) was created to implement the calculation using evaluated nuclear reaction cross section data.  The program has been compared to data measured from the irradiation of an iron sample using the University of Birmingham cyclotron.  It was also used to model the radioactivity of a target irradiated to 100 \acrfull{dpa} at a range of beam energies in order to minimise both the irradiation time and radioactive waste.  The results show an increase in target activity of 3 orders of magnitude for a 25MeV proton beam when compared to a 10MeV proton beam.

2. Sensitisation of austenitic steels, by processes such as welding, depletes Cr from grain boundaries removing the corrosion resistant \ce{Cr2O3} passive layer.  Previous (experimental) work has shown that these steels, when doped with Pd or Ru, retain corrosion resistance at the grain boundary\cite{scrstainless}.  This work takes a step towards investigating whether or not these \acrfull{pgm}s deplete or saturate at the grain boundary while under irradiation and this may be studied with \acrshort{md} simulations.  A computer program was developed to fit interatomic potentials to experimental data and \acrshort{dft} generated data.  The properties of \acrfull{fcc} Fe are computed using \acrshort{dft}.  These data along with other properties and \acrshort{dft} generated data were used to derive interatomic potentials (with a dominant ZBL pair for close range collision interactions) for Fe-Pd.  These potentials are a step towards \acrshort{md} simulations of the irradiated grain boundaries.  
\end{abstract}