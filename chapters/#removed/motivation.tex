\section{Motivation}

I have often questioned the motivation behind this work, and at times it has been a challenge to see the wood through the trees.

Mass produced steels are not perfect repeating crystals, but are made up of small grains.  Chromium is added to steel to make stainless steel, and this is more resistant to corroding than steel.  When steel is in a nuclear reactor, it will have to perform in extreme conditions.

\begin{itemize}
\item high temperatures
\item strain cause by high pressures 
\item radiation damage and strains resulting from this damage
\item corrosive environments while in the reactor
\item corrosive environments out of the reactor (e.g. fuel cladding in storage)
\end{itemize}

Radiation damages the steel in a number of ways, including directly knocking atoms out of their places within the steel as well as changing the elements.  One example is a neutron reacting with an iron atom, transmuting it into a cobalt atom.

In time, the radiation causes the amount of Chromium on the surface of the grains to drop, and as it falls the steel loses it's protection from corrosion at the boundary between the grains it is made of.  

This work is divided into two parts.

\subsection{Part 1: Activity Computer Program}

The materials must be tested before being used in a nuclear reactor.  One way to do this would be to place samples of the steel into a test reactor.  This is expensive, and as a byproduct the steel sample becomes radioactive.  It is difficult to create a large number of neutrons, but it is much easier, and cheaper, to create a beam of protons.  Protons can be accelerated in a machine such as the Cyclotron at the University of Birmingham.  

The damage that protons cause to a sample is not precisely the same, but it is a cheaper alternative to using neutrons.  One side effect that proton irradiation shares with neutron irradiation is the creation of radioactive waste.  

The first part of this work investigates how radioactive the samples become.  An existing set of equations, named after Mathematician Harry Bateman, were modified, and a computer program was created.  The user inputs what the material is, and the irradiation settings, and the program estimates how radioactive the sample will be.


\subsection{Part 2: Palladium-Iron Potential}

Adding Chromium to make stainless steel is not the only way to make a steel that is resistant to corrosion.  Adding metals such as Molybdenum and Palladium to steel can increase the resistance to corrosion, but Molybdenum is several hundred times the cost of iron ore, and Palladium is thousands of times as expensive.

Simulating radiation damage using a computer is now a feeasible and sensible way to investigate how these materials will be affected by radiation damage, and the simulations may reveal insights that experiments are not able to show.

Key to the simulations is being able to accurately calculate how the atoms interact with one another.  The second part of this work concentrates on deriving a mathematical description of how Palladium and Iron atoms affect one another, which would allow future simulations of steel with small amounts of Palladium added to it.

Radiation causes chrome to be depleted at the grain boundary.  If simulations show that Palladium is not depleted, it would suggest that the corrosion resistance is maintained, despite the decrease in the amount of chromium at the grain boundary due to radiation damage.








