




\subheadings{Martensitic Stainless Steel}

Unlike Austenitic and Ferritic, Martensitic stainless steels can be heat treated to harden the steel.  These steels are magnetic and have a FCT crystal structure.  They contain more than 10.5% Chromium and have a much lower Nickel than Austenitic grades, if any.  Two examples of such steels are ASME codes 410 and 431.  



\subheadings{Duplex Stainless Steel}




\subheadings{Austenitic Stainless Steel}


Austenite is a FCC allotrope of Iron, and austenitic stainless steels are useful in many applications, including a structural material for nuclear plant components, due to their resistance to corrosion.  In addition to 11 wt% or more chromium they require an austenite stabilising element such as carbon and/or nickel to be added (10).

Two examples of such steels are ASME codes 304 and 316.  Both have a high Chromium content, in the region of 18-20%, which is in excess of the minimum passive film requirement of around 10-11%.  The natural structure of such an Fe Cr alloy would be BCC, however 304 and 316 Steels contain Nickel (approximately 8% and 10% respectively) which is an Austenite stabiliser, and this give the FCC structure of the steel.  The 316 grade contains a minimum of 2% Molybdenum to improve its resistance to corrosion.



\subsection{Austenitic Stainless Steels in the Nuclear Industry}

Components of the Sizewell B PWR were constructed using stainless steels.  In particular, all the major parts of the reactor vessel were made from stainless steel, with the reactor coolant piping loop being made from austenitic stainless steel.  The majority of the components of the reactor coolant pump are also made from austenitic stainless steel as well as cladding inside the carbon steel pressure vessel. \cite{sizewellbdescription}

\cite{eprreactoroverview}

The Westinghouse AP1000 will also heavily rely on these steels.
\cite{ap1000preconstruction}

Low cobalt content



\subsection{Issues Associated with Austenitic Stainless Steels}


\subsection{Sensitization of Austenitic Stainless Steels}






When steels with Chromium content are heated, during processes such as welding, the metal undergoes Chromium sensitization.  On heating to temperatures between 600-700oC for many hours Chromium carbides, of the form M23C6, are created, depleting Chromium at the grain boundary.

Steel held at elevated temperatures
Chromium carbides precipitate at grain boundary
Depletes the grain boundary of chromium
Models of Tedmon et al, Fullman or Stawstrom and Hillert
Difficult to model, depends on grain structure, intergranular carbide spacing

\subsection{Sensitization of Austenitic Stainless Steels}

The Chromium in stainless steel only gives the steel its corrosion resistant properties if the content does not drop locally below the threshold percentage required.  When austenitic stainless steels are heated, chromium reacts with carbon at the grain boundary to form chromium carbide precipitates.  This, coupled with the slow diffusion rate of chromium from within the grain to the surface, removes the protection from the grain boundary.  This process is sensitization, and is a problem when welding steels ???.



\subsection{Intergranular Stress Corrosion Cracking}









