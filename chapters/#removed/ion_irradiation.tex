\section{Ion Irradiation}




\subsection{Emulating Neutron Radiation}



reference literature neutron damage

Neutrons 



\subsection{Proton Activation}

A major side effect from the process of nuclear fission is the creation of both radioactive fission fragments and radioactive isotopes within the components and stuructural material of the reactor.  Low energy Protons are not captured as low energy Neutrons would be, due to the opposing force between the proton and nucleus of the target atom.  Once the proton energies exceed a few MeV, they have sufficient energy to transmute target nuclei.

\subsubsection{Evaluated Nuclear Data Files}

Engineers and Scientists working for or researching in the nuclear industry need accurate data for a wide range of behaviours and properties of isotopes.  There is no magic formula to return the requested data for a given isotope, and this is why there is a need for ENDF files.  Experimental data 


\subsubsection{PADF}

The Proton Activation Data File was released in 2007 and contained nuclear reaction data for 2355 target nuclei, ranging from Magnesium (12) to Radon (86) with proton energies up to 150MeV.


\subsubsection{TALYS}

TALYS is a computer code, written for Linux and Unix systems, that is used to predict and analyse nuclear reactions.  It is also used as a tool to generate nuclear data.

\subsubsection{TENDL}

TENDL is a collection of files, each in the ENDF format, of nuclear reaction data generated by the TALYS code.







