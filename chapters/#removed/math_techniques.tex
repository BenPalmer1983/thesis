\section{Math Techniques}

\subsection{Lagrange Polynomial Interpolation}

The functions and functional that constitute the potential are represented in computer codes as a table of values.  The values are discrete points that are evenly spaced, and Lagrange polynomials are used to interpolate values.

\eqLagrangeDataSet

\eqLagrangePolynomialNew

\begin{lstlisting}[style=sPseudo,caption={Add two numbers function}]
// lagrange polynomial interpolation
function lpinterp_y(x, data)
  // x - point value being interpolated at to calc f(x)
  // data - four x,y data points (n by 2 array)
  n = len(data, 1)
  y = 0.0
  for i = 1,n
    l = 1.0
    for j = 1,n
      if (i != j) then
        l = l * (x - data[j][1]) / (data[i][1] - data[j][1])
      end if
    next j
    y = y + l * data[i][2]
  next i
end function lpinterp_y
\end{lstlisting}



There are typically hundreds or thousands of data points, so to interpolate, the closest few points are used.  Throughout the computer code four point interpolation was the prefered method, to balance computational speed with a well fitting polynomial.  

The gradient of the potential functions are also computed using lagrange polynomials.  The equation used is given below

\eqLagrangePolynomialDerivativeNew


\begin{lstlisting}[style=sPseudo,caption={}]
// lagrange polynomial interpolation
function lpinterp_dydx(x, data)
  // x - point value being interpolated at fo calc f'(x)
  // data - four x,y data points (n by 2 array)
  n = len(data, 1)
  dydx = 0.0
  for i = 1,n
    fx = 1.0
    gx = 1.0
    for j = 1, n
      if (i != j) then
        fx = fx / (data[i][1] - data[j][1])
        psum = 1.0
        for k = 1, n
          if (i != k and j != k) then
            psum = psum * (x - data[k][1])
          end if
        next k
        gx = gx + psum
      end if
    next j
    dydx = dydx + fx * gx * data[i][2]
  next i
end function lpinterp_dydx
\end{lstlisting}





\subsection{Splines}



\subsection{Simulated Annealing}



\subsection{Levenberg-Marquardt Optimisation}

