
\section{Ion Beam Induced Radioactivity} 

\subsection{Transmutation of Nuclei by Ion Irradiation}

Considering a simplified nuclear potential well, energetic protons approaching a nucleus may overcome the Coulomb potential barrier.  They are captured by the nucleus and held within the potential well by the strong nuclear force.  This process may leave the nucleus in an excited and unstable state, depending on the input energy of the proton and configuration of nucleons.  The process is probabilistic, and the average chance of a reaction (the microscopic cross section) may be measured as a function of the projectile, projectile energy and target, either experimentally or by optical model potential calculations.  The reaction rate is calculated from the microscopic cross section using the following equation:

\eqReactionRate

\begin{itemize}
	\item R	Reaction Rate (reactions per second)
	\item J	Beam current (A)
	\item \ensuremath{n_t}	Number density of target (atoms per cubic metre)
	\item \ensuremath{\sigma}	Microscopic reaction cross section (barns)
	\item e	Elementary charge (1.602177E-19C)
	\item \ensuremath{\delta}T	Target thickness (m)
\end{itemize}

\subsection{Radioactive Decay}

Radioactive decay is the random change in nucleons or energy state of an unstable nucleus.  It is impossible to predict when a single nucleus will decay, but the decay of a collection of nuclei is statistical in nature.  The radioactivity and number of unstable nuclei at time t can be predicted using the decay constant, \textlambda, for the radioactive isotope.  This constant is defined as follows:

\eqDecayConstant

The number of radioactive nuclei N(t) at time t is given by the following equation, where N(0) is the starting number of nuclei:

\begin{equation}
N(t) = N(0) \exp(-t \lambda)
\end{equation}

The activity A(t) of the radioactive nuclei is predicted at time t by using the following equations, where N'(t) is the change in amount of nuclei with respect to time:

\begin{equation}
A(t) = -N'(t) = \lambda N(t)
\end{equation}
\begin{equation}
A(t) = \lambda N(0) \exp(-t \lambda)
\end{equation}

\subsection{Bateman Equation for Radioactive Decay}

\begin{figure}[!h]
	\centering
	\begin{tikzpicture}[node distance=2cm]
	\node (parent) [startstop] {Parent Isotope, N$_{\text{1}}$(t)};
	\node (isotope1) [process, below of=parent] {1st Unstable Daughter Isotope, N$_{\text{2}}$(t)};
	\node (isotope2) [process, below of=isotope1] {2nd Unstable Daughter Isotope, N$_{\text{3}}$(t)};
	\node (stable) [startstop, below of=isotope2] {Stable Daughter Isotope, N$_{\text{4}}$(t)};
	%% arrows
	\draw [->] (parent) -- (isotope1);
	\draw [->] (isotope1) -- (isotope2);
	\draw [->] (isotope2) -- (stable);
	\end{tikzpicture}
	\captionsetup{font={it}}
	\caption{An example decay chain from an unstable parent isotope, through unstable daughter isotopes ending with a stable daughter isotope.}
	\label{fig:decaychain}
\end{figure}

The English mathematician Harry Bateman derived an equation (\ref{eq:bateman}) to calculate the amount of each isotope in a decay chain, illustrated in Figure \ref{fig:decaychain}, at time t.

\eqBateman

When a radioactive isotope decays, there may be more than one mode of decay, and this leads to branching factors.  Pb-214 only decays via beta decay to Bi-214, giving a branching factor of 1.0, whereas Bi-214 has a 99.979\% chance of decaying to Po-214 by beta decay and a 0.021\% of emitting an alpha particle and decaying to Tl-210 (branching factors of 0.99979 and 0.00021 respectively) \cite{jeff311}.

When a target material is irradiated, there is a source term for transmuted nuclei due to the irradiation.  The daughter isotopes of these transmuted isotopes will also be affected by the irradiation and will transmute further, giving a source term for each daughter isotope as a result of the irradiation.  Sources for each isotope in the decay chain, and branching factors between a parent isotope and its daughter isotope/s must be accounted for.


