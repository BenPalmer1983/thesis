\section{Magnetism}


\subsection{Brief History of Magnetism}

Magnetic minerals have been known to and used by civilzations for millenia.  Magnetite (lodestone) is a naturally occuring iron mineral.

In the 1800s, the link between between electricity and magnetism was explored, beginning with experimental work by Oersted and Faraday, eventually leading to Maxwell's equations.



\subsection{Magnetism: Moving Charges and Spin}

We are a product of our environment, and that is made all the more clear in the way our intuition led scientific development until relatively recently.  General Relativity and Quantum Mechanics were born in the first part of the 20th Century, and they describe the very large and very small realms we do not notice in every day life.

In describing how magnetism comes to be, it may be useful to use classical analogies, but that's all they are.

When a charged particle moves, it creates a magnetic field.  The field at some point j created by a charge moving at $\vec{v}$ at point i is calculated:

\begin{equation}
\begin{split}
\vec{B} = \frac{\mu_0 q}{4 \pi r_{ij}^3} \vec{v} \times \vec{r}_{ij}
\end{split}
\label{eq:eqSplineThreeEquations}
\end{equation}

Electrons do not orbit around a nucleus in the classical sense, but they do have an orbital angular momentum.  The overall magnetic field created by an atom is made up of:

\begin{itemize}
  \item orbital angular momentum of the electron
  \item spin of electrons
  \item spin of nucleons 
\end{itemize}

The contribution of the nucleus to the magnetic field is small, and the orbital and spin of electrons are mostly responsible.

\begin{equation}
\begin{split}
\vec{m_z} = \mu_B m_l \\
\text{where} m_l = 0, +-1, +-2, ... \\
\text{and} \mu_B = \frac{e \hbar}{2 m_e}
\end{split}
\label{eq:eqSplineThreeEquations}
\end{equation}


Spin is a property of quantum mechanics and it doesn't have an equivalent in classical mechanics, although it may be thought of similar to a spinning top or a rotating planet.  Electrons are point like particles, and do not actually rotate, but they do have an intrinsic angular moment (spin).  Where a spinning top might have a range of angular velocities, electrons have a quantised value and, as fermions, they have half integer spin.  


\begin{equation}
\begin{split}
f(x)
\end{split}
\label{eq:pauliMatrices}
\end{equation}







The Bohr magneton is a unit for measuring magnetic moment.

$1$ Bohr magneton $ = 5.79 eV/T$


\subsubsection{Electron Motion}

The motion of an electric charge causes a magnetic field, and this is a consquence of special relativity.  

\eqAmpereLawIntegral



\subsection{Ferromagnetism and Antiferromagnetism}

\subsubsection{Hund's Rule}

Electrons fill the shells of a ground state atom such that the energy is minimised.  Four quantum numbers are used to describe electrons bound to an atom: n, l, $m_l$ and $m_s$.  The Pauli exclusion principle states that the electrons, which are fermions, cannot have the same quantum numbers as another electron bound to that atom.

Electrons may be spin up or spin down so, two electrons can have the same n, l, $m_l$ with two $m_s$ "slots" available.  The energy is minimised by not pairing up and down electrons until all the free "slots" due to the coulomb interaction between the electrons.  Once each slot contains one electron, they begin to pair.

The electronic configuration of Iron highlights this:

\begin{equation}
\begin{split}
&\text{1s2   } \underline{\uparrow \downarrow} \\
&\text{2s2   } \underline{\uparrow \downarrow} \\
&\text{2p6   } \underline{\uparrow \downarrow} \:\:\:\:  \underline{\uparrow \:\:} \:\:\:\:  \underline{\uparrow \:\:} \\
&\text{3s2   } \underline{\uparrow \downarrow} \\
&\text{3p6   } \underline{\uparrow \downarrow} \:\:\:\:  \underline{\uparrow \:\:} \:\:\:\:  \underline{\uparrow \:\:} \\
&\text{4s2   } \underline{\uparrow \downarrow} \\
&\text{3d7   } \underline{\uparrow \downarrow} \:\:\:\:  \underline{\uparrow \:\:} \:\:\:\:  \underline{\uparrow \:\:} \:\:\:\: \underline{\uparrow \:\:} \:\:\:\:  \underline{\uparrow \:\:}
\end{split}
\label{eq:eqSplineThreeEquations}
\end{equation}

Rather than fill the 3d shell from left to right, the first five slots take electrons with spin in the same direction, and the remaining two electrons fill two of the slots with spin opposite to the first five of the shell.

