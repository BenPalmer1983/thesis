\section{Force Matching}

To derive a potential, one may approach the problem from first principles in an attempt to replicate reality.  It has been more useful, however, to lose any physical elegance \cite{twobandackland} to give potentials that work for specific elements under certain conditions.  Force data, gathered experimentally or by first-principles calculations, has been used to develop potentials since the 1990s.  The force matching method was developed in 1994 by Ercolessi and Adams \cite{forcematchingmethod} to link the more accurate, more processor and memory intensive, world of first-principles calculations to Molecular Dynamics.

The force-matching method uses the difference between the actual force (either measured experimentally or calculated by first-principles calculations) 

Given a set of M different atomic configurations, and a potential with a set of L parameters ($ \vec{p} $), the function $Z_F$ is a measure of the difference between the the forces calculated using the potential for all configurations and the actual (or DFT generated) forces.

\begin{equation}
\begin{split}
Z_F(\vec{\alpha}) = \sum _{k=1}^M \sum _{i=1} ^{k} \sum _{j=1} ^{3} \lvert \vec{F^k_{i,j} (\vec{\alpha})} - \vec{F^0_{i,j}} \rvert^2
\end{split}
\label{eq:eqForceMatchingB}
\end{equation}

This may be extended to include the calculation of other properties, including the cohesive energy of atoms, the lattice parameter, bulk modulus, elastic constants and so on.  Each may be weighted depending on how important the property is to the simulation the potential is required for.

\begin{equation}
\begin{split}
Z(\vec{p}) = w_{F} Z_F(\vec{p}) + w_{b0} Z_{b0}(\vec{p}) + w_{e0} Z_{e0}(\vec{p}) + w_{a0} Z_{a0}(\vec{p}) + w_{ec} Z_{ec}(\vec{p}) + w_{ecoh} Z_{ecoh}(\vec{p})
\end{split}
\label{eq:eqForceMatchingA}
\end{equation}

