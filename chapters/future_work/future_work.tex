\chapter{Future Work}
\label{chapter:futurework}

\begin{changemargin}{1.0cm}{1.0cm} 
\abstractpreamble{During the development of the Activity code, a number of simplifications were made.  In future, some of these could be examined more closely to determine how much impact the simplifications have on the end result.  Other simplifications could also be applied, again once their significance has been determined, in order to make the calculation less computationally expensive.\\
\\
Two potentials were developed in this work, but they have not been used for the purpose they were created for.  There is room to improve these potentials by providing better fitting data and by continuing the fitting process for longer.  They need to be tested in terms of their ability to model damage cascades and then put into practice using either \acrshort{md} or \acrshort{akmc}.}
\end{changemargin}


\section{Activity Code}

\subsection{Experimental Activity Readings for Ion Irradiated Targets}

The Activity code was developed using data from the \acrshort{tendl} data file.  To test the validity of the code, an Iron sample was irradiated and its activity was measured several days after irradiation.

To validate the code further, a range of pure targets and alloys would be irradiated by a cyclotron, with targets of varying thickness and beams of varying fluences and energies.  This would generate data to be used to compare to the results predicted by the code.  


\subsection{Code Improvements}

A number of improvements could be made to the code.  While Python is a popular language, and a feature rich one, there are still benefits in using a language such as Fortran.  This is especially true when coupled with OpenMP.  There certainly are parts of the code that could be improved by using F2PY to decrease runtime.

All versions of the code have relied on data from \acrshort{srim}.  It might be useful to integrate or build an ion transport class so a calculation may be run once, rather than having to first generate the transport data on one platform and run the Activity code on another.  

It should also be investigated whether or not a history of many particles is needed.  There are already average energy stopping tables.  If there is little difference in calculated activity when using the histories of a thousand ions and energy stopping tables, then the stopping tables should be used.  It would remove the need to generate the ion history, which in itself is a long process, and it would speed up the computation of target activity.

The energy range of cross section values and particle types could be extended to higher energies and a wider range of particles including deuterons, tritons and helium ions.  This would require more time but would be useful in the case of the University of Birmingham cyclotron as it is capable of accelerating these particles as well as protons.

Currently the code does not take into account 511KeV gammas created by the annihilation of positrons from beta+ decay or pair production from higher energy photons.  This will affect the calculations during irradiation and during the cooling period.  Where the transmuted nuclei is left in an excited state due to the energy difference, the gammas that will be released are not yet accounted for.  This is more important during irradiation and may not need to be considered providing there is adequate shielding whilst the target is in the ion beam.

As the target becomes thicker, the orientation of the irradiated target to the end user handling it will become more important.  Incorporating this into the end calculation to display the gamma fluence radially in 2 dimensions from the target would help the user position themselves relative to the irradiated face of the target to minimise the dose received.

Whilst the target is being irradiated, there will be a high number of gammas as well as other radiation such as neutrons, protons, electrons and other light ions.  It has always been assumed that this process will be carried out behind adequate shielding as it is only a danger when the proton beam is on.  When out of the beam the primary concern to a person nearby the target are the gammas released by radioactive decay.  There may also be reasons to consider the amount of beta and alpha radiation emitted by the target, although this might only be a concern if the target is to be machined, releasing ingestible radioactive dust, or handled without protection where the beta particles will be in a position to travel through skin.  Ideally, both these circumstances would be avoided, but the utility could be added to the code.

More features could be added including a choice of dose units and whether or not the dose is calculated for a given volume of space or for a given person with certain parameters.



\section{Potential Fitting and MD Simulations}

\subsection{Larger Supercells}

In many examples in the literature, the supercell sizes were at least 4x4x4 containing 256 atoms for \acrshort{fcc} crystals.  These require much more memory and processing power, especially considering the lack of symmetry and inclusion of magnetism in the calculations.  

If these potentials were to be improved in the future, all the calculations would contain 128 atoms for \acrshort{bcc} and \acrshort{hcp} configurations and 256 atoms for \acrshort{fcc}.  Ideally the smearing would be reduced further from the value used here, of 0.04, and the number of k-points increased within reason.

This higher number of atoms would have the benefit of increasing the number of data points used to fit the potentials, as a consequence of increasing the number of force values.  It would also allow the 1\% doped value to be better represented by using 1, 2 or 3 atoms per 256 (0.4\%, 0.8\%, 1.2\%) rather than 1 per 32 atoms (3\%).

\subsection{Alloy Bulk Properties}

The original potentials used the bulk properties of Fe, Pd and Ru.  The calculations may then be performed using the QEEOS code to compute the bulk properties of Fe-Pd and Fe-Ru alloys.  As the arrangement of atoms would not be unique it would not be as useful to use the figures output by QEEOS, but it would be more useful to use the files output by Quantum Espresso.  

\subsection{Defects}

In the derivation of the potentials in this work, more weighting was given to bulk and slab/surface configurations.  There were also issues in \acrshort{scf} convergence for a number of the defect configurations.  More effort could be put towards investigating how to help PWSCF achieve convergence.  In addition a number of the defects in Fe could be self interstitials as well as \acrshort{pgm}s.

\subsection{MEAM for FCC Iron}

With a collinear spin-polarized \acrshort{dft} calculation, the relaxed structure was slightly longer in the y direction.  Investigating the use of an angularly dependent potential might be worthwhile and the \acrshort{meam} type potential with an angularly dependent electron density would be a starting point (section \ref{section:meam}).

\subsection{Improved Potential: Fe-Ni-Cr-Pd}

Given more time and a larger computer, more complicated \acrshort{dft} calculations could be performed that better represent the material, austenitic stainless steel.  As already mentioned, there are drawbacks (in terms of memory and computational time) and benefits in using a supercell with more atoms.  For example a 1\% \acrshort{pgm} doped steel could be represented as a 4x4x4 supercell containing 256 atoms.  For \gls{304SS} approximately 50 Cr, 25 Ni, 178 Fe and 3 Pd or 3 Ru atoms would be needed (and small variations on these figures).

\subsection{PKA Cascade Trials in LAMMPS or DL\_POLY}

The first set of trials using the potential should be \acrshort{pka}s within a suitably sized simulation box, perhaps tens of thousands to hundreds of thousands of atoms.  Individual cascades may be simulated, examined and compared to previous simulations as discussed earlier in this work.
 

\subsection{Radiation Induced Segregation Trials}

The ultimate goal for this part of the work is to determine whether or not \acrshort{pgm}s are depleted at the grain boundary removing the protection to corrosion that they provide.  There is a possibility of using the potentials with molecular dynamics, but the time period to cover the number of displacements required may be too large for classical molecular dynamics.  An alternative use for the potential would be to explore \acrfull{akmc} and this could allow simulation over a much longer period of time in order to reach damage doses up to 100 \acrshort{dpa}.



