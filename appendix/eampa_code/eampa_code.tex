\chapter{EAMPA Code}
\label{chapter:appendix-eampa}
 
The full code is available at https://github.com/BenPalmer1983/eampa

Several of the important functions and subroutines that make up the code are listed below. 



\section{Energy, Force, and Stress Subroutine}
\label{section:appendixenergyforcestress}

\lstinputlisting[style=sFortran,caption={Energy, Force and Stress subroutine (Fortran)},label={listing:efssubroutine}]{appendix/eampa_code/fortran/atom.efs.f90}



\section{Neighbour List Subroutine}

\lstinputlisting[style=sFortran,caption={Energy, Force and Stress subroutine (Fortran)},label={listing:efssubroutine}]{appendix/eampa_code/fortran/atom.nl.f90}



\section{Fortran Programmed Functions}

\lstinputlisting[style=sFortran,caption={Polynomial Embedding (Fortran)},label={listing:efssubroutine}]{appendix/eampa_code/fortran/fnc.polynomial_embedding.f90}

\lstinputlisting[style=sFortran,caption={Polynomial Embedding Gradient (Fortran)},label={listing:efssubroutine}]{appendix/eampa_code/fortran/fnc.polynomial_embedding_grad.f90}

\lstinputlisting[style=sFortran,caption={Spline (Fortran)},label={listing:efssubroutine}]{appendix/eampa_code/fortran/fnc.spline.f90}



\section{Main fitting class}

\lstinputlisting[style=sPython,caption={Fitting Class (Python)},label={listing:efssubroutine}]{appendix/eampa_code/python/potfit.py}



\section{Global Fitting Algorithms}
\label{section:globalfittingalgorithms}

\label{section:simulatedannealing}
\lstinputlisting[style=sPython,caption={Simulated Annealing Class (Python)},label={listing:efssubroutine}]{appendix/eampa_code/python/simulated_annealing.py}

\label{section:geneticalgorithm}
\lstinputlisting[style=sPython,caption={Genetic Algorithm Class (Python)},label={listing:efssubroutine}]{appendix/eampa_code/python/genetic_algorithm.py}




\section{Bulk Properties}
\label{section:appendixbpaddconfiguration}

\lstinputlisting[style=sPython,caption={Bulk Properties Class (Python)},label={listing:efssubroutine}]{appendix/eampa_code/python/bp.py}





\section{Interpolation}
\label{section:interpolationpseudo}

Lagrange polynomial interpolation was programmed in Fortran and was used in a number of places throughout the program (listing \ref{listing:lagrangepoly}).

\lstinputlisting[style=sFortran,caption={Lagrange polynomial interpolation},label={listing:lagrangepoly}]{appendix/eampa_code/fortran/interp.interpn.f90}


\section{Interpolation (Gradient)}
\label{section:interpolationgradientpseudo}

A modified version of the Lagrange polynomial interpolation was programmed in Fortran to interpolate the derivative of a set of data points at a given position between (and including) the start and end point (listing \ref{listing:lagrangepolygradient}).

\lstinputlisting[style=sFortran,caption={Lagrange polynomial interpolation (Gradients)},label={listing:lagrangepolygradient}]{appendix/eampa_code/fortran/interp.interpndydx.f90}
