\begin{lstlisting}[style=sPseudo,caption={Simple simulated annealing subroutine}]
// Simulated Annealing
subroutine simulated_annealing(f, p, pv, x, y, t, p_best)
  // f -       (IN)  the function f(p, x) for which the parameters are being optimised
  // p -       (IN)  array containing parameter/parameters
  // pv -      (IN)  array of maximum parameter/s variance
  // x -       (IN)  array of x points
  // y -       (IN)  array o fy points
  // t -       (IN)  starting temperature
  // p_best -  (OUT) optimised parameters
  //

  rss = (f(x[:], p[:]) - y[:])**2 
  
  rss_best = rss
  p_best[:] = p[:]
  
  // These may be modified as required
  outer_loops = 100
  inner_loops = 1000
  t_decrease = 0.99
  pv_decrease = 0.99
  
  for outer = 1, outer_loops
    for inner = 1, inner_loops
    
      r[:] = 0.5 - random_float(0.0, 1.0)  // array of random floats (same size as p)
      
      pt[:] = p[:] + r[:] * pv[:]
      rss = (f(x[:], pt[:]) - y[:])**2 
      
      // If better, always use new parameters
      if (rss < rss_best) then
        rss_best = rss
        p[:] = pt[:]
        p_best[:] = pt[:]
        
      // If worse, sometimes use parameters (based on how good they are, and the temperature)
      else      
        // as t -> 0   exp((rss_best - rss)/t) -> 0
        // as (rss_best - rss) -> 0    exp((rss_best - rss)/t) -> 1  (where t = 1) 
        rn = random_float(0.0, 1.0)
        if (rn <= exp((rss_best - rss) / t)) then
          p[:] = pt[:]
        end if
      end if     
      
    end for
    
    // Decrease temperature
    t = t_decrease * t
    
    // Decrease variance
    pv[:] = pv_decrease * pv[:]
    
  end for
end subroutine
\end{lstlisting}
