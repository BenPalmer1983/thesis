\begin{lstlisting}[style=sPseudo,caption={Add two numbers function}]
// lagrange polynomial interpolation
function lpinterp_y(x, data)
  // x - point value being interpolated at to calc f(x)
  // data - four x,y data points (n by 2 array)
  n = len(data, 1)
  y = 0.0
  for i = 1,n
    l = 1.0
    for j = 1,n
      if (i != j) then
        l = l * (x - data[j][1]) / (data[i][1] - data[j][1])
      end if
    next j
    y = y + l * data[i][2]
  next i
end function lpinterp_y


// lagrange polynomial interpolation
function lpinterp_dydx(x, data)
  // x - point value being interpolated at fo calc f'(x)
  // data - four x,y data points (n by 2 array)
  n = len(data, 1)
  dydx = 0.0
  for i = 1,n
    fx = 1.0
    gx = 1.0
    for j = 1, n
      if (i != j) then
        fx = fx / (data[i][1] - data[j][1])
        psum = 1.0
        for k = 1, n
          if (i != k and j != k) then
            psum = psum * (x - data[k][1])
          end if
        next k
        gx = gx + psum
      end if
    next j
    dydx = dydx + fx * gx * data[i][2]
  next i
end function lpinterp_dydx


function lpinterp(x, data, d=0)

if (d == 0) then
  return lpinterp_y(x, data, n)
else if (d == 1) then
  return lpinterp_dydx(x, data, n)  
else if (d > 1) then
  n = len(data, 1)
  dcount = 1
  // Recusively build a f', f'', f''' etc as required
  do while(dcount < d) 
    for ncount = 1, n
      data_temp[ncount][1] = data[ncount][1]
      data_temp[ncount][2] = lpinterp_dydx(x,data[ncount][1])    
    next ncount
    data = data_temp
    dcount = dcount + 1
  end do
  return lpinterp_dydx(x, data, n)    
end if
end function lpinterp

\end{lstlisting}