%%==================================================================================================
%%==================================================================================================

\newcommand{\eqGaussLaw}{
\begin{equation}
\begin{split}
%%
\nabla \cdot \vec{D} = \rho_V
%%
\end{split}
\label{eq:eqGaussLaw}
\end{equation}
}


\newcommand{\eqGaussMagnetismLaw}{
\begin{equation}
\begin{split}
%%
\nabla \cdot \vec{B} = 0
%%
\end{split}
\label{eq:eqGaussMagnetismLaw}
\end{equation}
}


\newcommand{\eqFaradayLaw}{
\begin{equation}
\begin{split}
%%
\nabla \times \vec{E} = - \frac{\delta \vec{B}}{\delta t}
%%
\end{split}
\label{eq:eqFaradayLaw}
\end{equation}
}



\newcommand{\eqAmpereLaw}{
\begin{equation}
\begin{split}
%%
\nabla \times \vec{H} = \frac{\delta \vec{D}}{\delta t} + \vec{J}
%%
\end{split}
\label{eq:eqAmpereLaw}
\end{equation}
}




\newcommand{\eqAmpereLawIntegral}{
\begin{equation}
\begin{split}
%%
\oint \vec{B} \cdot dl = \mu_0 I
%%
\end{split}
\label{eq:eqAmpereLawIntegral}
\end{equation}
}








%%==================================================================================================
%%==================================================================================================





