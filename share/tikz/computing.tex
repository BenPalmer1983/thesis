\newcommand{\cpuDiagram}{
  \begin{tikzpicture}
  %% Background	
  \draw [black, fill=white] (6,0.5) rectangle (12,8.0);
  %%\draw [fill=white] (2,0) rectangle (3.5,1);


%%\node at (1,1) {yes};
  \end{tikzpicture}  
	
}   




%%%%%%%%%%%%%%%%%%%%%%%%%%
% Programming: Conditional
%%%%%%%%%%%%%%%%%%%%%%%%%%

\newcommand{\conditionalChartA}{
  \resizebox{8.0cm}{!}{
%% Draw boxes
    \begin{tikzpicture}[node distance=2.4cm]
% Row 1
    \node (codeStart) [startstop] {Code Starts};
% Row 2
    \node (condA) [decision, below of=codeStart] {Conditional A};
% Row 3
    \node (condAProcessA) [process, below of=condA, xshift=-3cm] {True: execute code in IF block};
    \node (condAProcessB) [process, below of=condA, xshift=3cm] {False: execute code in ELSE block};
% Row 4
    \node (condB) [decision, below of=condAProcessA, xshift=3cm] {Conditional B};
% Row 5
    \node (condBProcessA) [process, below of=condB, xshift=-3cm] {True: execute code in IF block};
    \node (condBProcessB) [process, below of=condB, xshift=3cm] {False: execute code in ELSE block};
% Row 6
    \node (codeEnd) [startstop, below of=condBProcessA, xshift=3cm] {Code Ends};
%% Draw arrows
% Row 1-2
    \draw [thick,->] (codeStart) -- (condA);
  %Row 2-3
    \draw [thick,->] (condA) -- (condAProcessA);
    \draw [thick,->] (condA) -- (condAProcessB);
  %Row 3-4
    \draw [thick,->] (condAProcessA) -- (condB);
    \draw [thick,->] (condAProcessB) -- (condB);
  %Row 4-5
    \draw [thick,->] (condB) -- (condBProcessA);
    \draw [thick,->] (condB) -- (condBProcessB);
  %Row 5-6
    \draw [thick,->] (condBProcessA) -- (codeEnd);
    \draw [thick,->] (condBProcessB) -- (codeEnd);
    \end{tikzpicture}
  }
}



\newcommand{\conditionalChartB}{
  \resizebox{8.0cm}{!}{
  %% Draw boxes
  \begin{tikzpicture}[node distance=2.4cm]
% Row 1
  \node (codeStart) [startstop] {Code Starts};
% Row 2
  \node (condA) [decision, below of=codeStart] {Conditional A};
% Row 3
  \node (condAProcessA) [process, below of=condA, xshift=-3cm] {IF A};
  \node (condAProcessB) [process, below of=condA, xshift=-3cm] {ELIF B};
  \node (condAProcessC) [process, below of=condA, xshift=-3cm] {ELIF C};
  \node (condAProcessD) [process, below of=condA, xshift=3cm] {ELSE};
% Row 4
  \node (codeEnd) [startstop, below of=condBProcessA, xshift=3cm] {Code Ends};
  %% Draw arrows
  % Row 1-2
  \draw [thick,->] (codeStart) -- (condA);
  %Row 2-3
  \draw [thick,->] (condA) -- (condAProcessA);
  \draw [thick,->] (condA) -- (condAProcessB);
  \draw [thick,->] (condA) -- (condAProcessC);
  \draw [thick,->] (condA) -- (condAProcessD);
  %Row 3-4
  \draw [thick,->] (condAProcessA) -- (codeEnd);
  \draw [thick,->] (condAProcessB) -- (codeEnd);
  \draw [thick,->] (condAProcessC) -- (codeEnd);
  \draw [thick,->] (condAProcessD) -- (codeEnd);
  \end{tikzpicture}
  }
}








\newcommand{\twoDGrid}{
  \resizebox{12.0cm}{!}{
    \begin{tikzpicture}
    %% Background	
    \foreach \i in {1,...,5}
    {
      \foreach \j in {1,...,5}
      {
        %% Left hand box
        \tikzmath{\iPrint = int(\i - 1);}
        \tikzmath{\jPrint = int(\j - 1);}
        \tikzmath{\offset = 6.5;}
        \draw [black, fill=none] (\i-1,\j-1) rectangle (\i,\j) node[pos=.5] {\iPrint , \jPrint};

        %% Right hand box
        \tikzmath{\rd = sqrt((\j-1) * (\j-1) + (\i-1) + (\i-1));}
        \tikzmath{\rd = int(\rd * 100);}
        \tikzmath{\rdPrint = \rd / 100;}
        \draw [black, fill=none] (\i-1+\offset,\j-1) rectangle (\i+\offset,\j) node[pos=.5] {\pgfmathprintnumber{\rdPrint}};

        
      }
    }
    \end{tikzpicture}  
  }
}   


