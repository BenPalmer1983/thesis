\section{Reference Database}

In order to fit the potentials, it was necessary to build a reference database.  This consists of known experimental bulk properties along with forces, energies and stresses calculated using DFT.

32 atom randomised

108 atom defects

%%\begin{tikzpicture}
%%\fccTetraConfig{3}{3}{3}{0.06};  %% {width (x), depth(y), height(z)}
%%\end{tikzpicture}  

%%\begin{tikzpicture}
%%\fccOctaConfig{3}{3}{3}{0.06};  %% {width (x), depth(y), height(z)}
%%\end{tikzpicture}  


\subsection{DFT Calibration}

\subsubsection{Choice of Pseudopotentials}






\subsubsection{Energy and Charge Density Cutoffs}




\subsubsection{K-Points}





\subsubsection{nbnd}




\subsection{Atomic Configurations for DFT Calculations}

The interatomic potentials are designed to reproduce the forces, energies, stresses and bulk properties








