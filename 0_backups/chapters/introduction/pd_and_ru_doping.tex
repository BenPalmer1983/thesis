\section{Doping with Palladium, Ruthenium and other Platinum Group Metals}




\subsection{Palladium and Ruthenium at the Grain Boundary}




\section{Doping Alloys with Palladium and Ruthenium}

The corrosion resistant benefits of small amounts of palladium (4) and ruthenium (5) have been investigated.  Due to the high cost of both palladium and ruthenium, there has been much interest in finding the optimum percentage of both.  Ru and Pd doped stainless steels would be worth the additional cost if their corrosive resistant properties are maintained at the grain boundary.  In particular, ruthenium has been shown to be beneficial to stainless steels where chloride containing solutions are concerned (5).  There have been a number of proposed mechanisms of how PGMs enhance the corrosion resistance of steel.




\subsection{Cathodic Modification}

Cathodic modification is one method that has been known for the last century (6), and the inhibition of anodic dissolution of the stainless steel has also been studied (7).  The layer of PGM adatoms, atoms on the crystal surface, block anodic sites within the crystal which stops corrosion attack local to the adatoms (1).

\subsection{Effect on Intergranular Stress Corrosion Cracking}


\subsection{Other Notable Corrosion Resistance Enhancing Alloys}

Molybednum has been added to these steels to improve the resistance of pitting corrosion (4), and it has been shown to improve resistance against chloride containing solutions. Enhancing Resistance to Corrosion


