


%%====================================================================================================================================================
%%
%%  Part 2: Activity Code
%%
%%====================================================================================================================================================

\section{Activation by Ion Irradiation}







The Bateman equation was derived using Laplace transforms, and this same method has been used to develop a modified equation that incorporates branching factors and production rates for each isotope in the decay chain, as illustrated by Figure \ref{fig:decaytree}.

\begin{figure}[!h]
	\centering
	\begin{tikzpicture}[node distance=2cm]
	% Row 1
	\node (parent) [startstop] {Parent Isotope};
	\node (parent_source) [process, left of=parent, xshift=-4cm] {Parent Source};
	% Row 2
	\node (parent_branch) [decision, below of=parent] {Branching};
	% Row 3
	\node (daughter_1a) [process, below of=parent_branch, xshift=-2.5cm] {Daughter 1 (Branch A)};
	\node (daughter_1a_src) [process, left of=daughter_1a, xshift=-2.5cm] {External Source};
	\node (daughter_1b) [process, below of=parent_branch, xshift=2.5cm] {Daughter 1 (Branch B)};
	\node (daughter_1b_src) [process, right of=daughter_1b, xshift=2.5cm] {External Source};
	% Row 4
	\node (daughter_1a_branch) [decision, below of=daughter_1a, xshift=0.5cm] {Branching};
	% Row 5
	\node (daughter_2a) [process, below of=daughter_1a_branch, xshift=-0.5cm] {Daughter 2 (Branch A)};
	\node (daughter_2a_src) [process, left of=daughter_2a, xshift=-2.5cm] {External Source};
	\node (daughter_2b) [process, below of=daughter_1a_branch, xshift=4.5cm] {Daughter 2 (Branch B)};
	\node (daughter_2b_src) [process, right of=daughter_2b, xshift=2.5cm] {External Source};
	
	% arrows
	\draw [thick,->] (parent_source) -- (parent);
	\draw [thick,->] (parent) -- (parent_branch);
	\draw [thick,->] (parent_branch) -- (daughter_1a);
	\draw [thick,->] (parent_branch) -- (daughter_1b);
	\draw [thick,->] (daughter_1a_src) -- (daughter_1a);
	\draw [thick,->] (daughter_1b_src) -- (daughter_1b);
	\draw [thick,->] (daughter_1a) -- (daughter_1a_branch);
	\draw [thick,->] (daughter_1a_branch) -- (daughter_2a);
	\draw [thick,->] (daughter_1a_branch) -- (daughter_2b);
	\draw [thick,->] (daughter_2a_src) -- (daughter_2a);
	\draw [thick,->] (daughter_2b_src) -- (daughter_2b);
	
	%\draw [->] (isotope1) -- (isotope2);
	%\draw [->] (isotope2) -- (stable);
	\end{tikzpicture}
	\captionsetup{font={it}}
	\caption{An example of several decay chains including branching factors and possible external source terms for each isotope on each chain.}
	\label{fig:decaytree}
\end{figure}


\subsection{Laplace Transform}

Laplace Transforms (\ref{eq:eqLaplaceTransform}) are a useful mathematical tool, and allow ordinary differential equations to be solved by simple algebraic manipulation in the s domain.  Bateman took advantage of Laplace Transforms in deriving his equation, and this is the method that has been taken here as well.

\eqLaplaceTransform


\subsection{Constructing the Differential Equations}

The first step is to set up differential equations for the parent isotope, unstable daughter isotopes and stable daughter isotope.  The parent isotope has a source term, due to production, and a loss term, due to decay.  The unstable daughter isotopes have two source terms, from the production by irradiation induced transmutation and the decay of preceding isotopes in the decay chain, and a loss term, due to decay.  Finally, the stable daughter that finalizes the decay chain has two source terms (the same as the unstable daughters) but no loss term.

The variables (and vectors) used in these equations are defined as follows:
\begin{itemize}
	\item $\vec{\lambda}$  vector containing isotope decay constants $\lambda_i$
	\item $\vec{b}$  vector containing isotope to isotope branching factors $b_i$
	\item $\vec{w}$  vector containing isotope production rates $w_i$
	\item $t$  time at which activity/amount of isotope is measured
	\item $N_{i}(0)$ starting amount of the i\textsuperscript{th} isotope
	\item $N_{i}(t)$ amount of the i\textsuperscript{th} isotope at time t
	\item $N'_{i}(t)$ change in amount of the i\textsuperscript{th} isotope, with respect to time, at time t
\end{itemize}

The differential equations for the parent isotope (first isotope), unstable daughter isotopes (i\textsuperscript{th} isotopes) and stable, final, daughter isotope (zth isotope) in the time domain are as follows:

\begin{equation}
N'_{1}(t) = \omega_{1} - \lambda_{1} N_{1} (t)
\end{equation}

\begin{equation}
N'_{i}(t) = \omega_{i} + b_{i-1} \lambda_{i-1} N_{i-1} (t) - \lambda_{i} N_{i} (t)
\end{equation}

\begin{equation}
N'_{z}(t) =  \omega_{z} + b_{z-1} \lambda_{z-1} N_{z-1} (t)
\end{equation}

Applying the Laplace Transform to these three differential equations allows them to be manipulated and solved algebraically in the s-domain.

\begin{equation}
N_{1}(s) = \frac{1}{s+\lambda_{1}} N_{1}(0) + \frac{1}{s(s+\lambda_{1})} \omega_{1}
\end{equation}

\begin{equation}
N_{i}(s) = \frac{1}{s ( s+ \lambda_{i})} \left(\omega_{i} \right) + \frac{1}{s+ \lambda_{i}} \left( b_{i-1} \lambda_{i-1} N_{i-1} (s) \right) + \frac{1}{s+ \lambda_{i}} N_{i} (0)
\end{equation}

\begin{equation}
N_{z}(s) = \frac{1}{s^2} \omega_{z} + \frac{1}{s} b_{z-1} \lambda_{z-1} N_{z-1} (s) + \frac{1}{s} N_{z}(0)
\end{equation}


\subsection{Numerical Inversion of the Laplace Transform}

The Gaver-Stehfest\cite{stehfest} algorithm was developed in the 1960s and 1970s and is a method of calculating the inverse of a Laplace Transform in the real number domain.  It is an easy to implement and reasonably accurate method, although it is an approximation to the real value.  A comparison between an analytic and numeric inversion for the unstable isotope Po-218 is discussed at the end of this section (figure \ref{fig:po218decay}).

\begin{equation}
f(t) \approx f_{n}(t) = \frac{\ln(2)}{t} \sum_{k=1}^{2n} a_{k}(n)F(s) \textnormal{ where } n \ge 1, t>0
\end{equation}

\begin{equation}
s = \frac{k \ln(2)}{t}
\end{equation}

\begin{equation}
a_{k}(n) = \frac{(-1)^{(n+k)}}{n!} \sum_{j=Floor(\frac{k+1}{2})} j^{n+1} \left( \begin{matrix} n \\ j \end{matrix} \right)  \left( \begin{matrix} 2j \\ j \end{matrix} \right)  \left( \begin{matrix} j \\ k-j \end{matrix} \right)
\end{equation}

The equation for the i\textsuperscript{th} isotope may be calculated by recursively calculating the equations by numeric inversion, starting from the first (parent isotope) and inserting the result into each subsequent recursion until the i\textsuperscript{th} isotope is reached (changing the equations appropriately for the parent, unstable daugher and stable daughter isotopes).


\subsection{Analytic Solution by Partial Fraction Expansion}

The equation for the i\textsuperscript{th} isotope in the s domain can be written in full by substituting the preceding equation until the parent isotope is reached, and this full equation may be rearranged with the production amount of each isotope and starting amount of each isotope in individual terms.  Each of these terms is multiplied by a fraction that can be expanded, using partial fractions, and inverted analytically.

This is illustrated with an example unstable isotope, fourth in the decay chain (including the parent isotope):

\begin{equation}
\begin{split}
N_{4}(s) =
\frac{1}{(s+\lambda_1)(s+\lambda_2)(s+\lambda_3)(s+\lambda_4)} b_{2} b_{3} b_{4} \lambda_1 \lambda_2 \lambda_3 N_{1}(0) \\
+ \frac{1}{(s+\lambda_2)(s+\lambda_3)(s+\lambda_4)} b_{3} b_{4} \lambda_2 \lambda_3 N_{2}(0) \\
+ \frac{1}{(s+\lambda_3)(s+\lambda_4)} b_{4} \lambda_3 N_{3}(0) \\
+ \frac{1}{(s+\lambda_4)} N_{4}(0) \\
+ \frac{1}{s(s+\lambda_1)(s+\lambda_2)(s+\lambda_3)(s+\lambda_4)} b_{2} b_{3} b_{4} \lambda_1 \lambda_2 \lambda_3 \omega_{1} \\
+ \frac{1}{s(s+\lambda_2)(s+\lambda_3)(s+\lambda_4)} b_{3} b_{4} \lambda_2 \lambda_3 \omega_{2} \\
+ \frac{1}{s(s+\lambda_3)(s+\lambda_4)} b_{4} \lambda_3 \omega_{3} \\
+ \frac{1}{s(s+\lambda_4)} \omega_{4}
\end{split}
\end{equation}

An example stable isotope, fourth (last) in the decay chain (including the parent isotope):

\begin{equation}
\begin{split}
N_{4}(s) =
\frac{1}{s(s+\lambda_1)(s+\lambda_2)(s+\lambda_3)} b_{2} b_{3} b_{4} \lambda_1 \lambda_2 \lambda_3 N_{1}(0) \\
+ \frac{1}{s(s+\lambda_2)(s+\lambda_3)} b_{3} b_{4} \lambda_2 \lambda_3 N_{2}(0) \\
+ \frac{1}{s(s+\lambda_3)} b_{4} \lambda_3 N_{3}(0) \\
+ N_{4}(0) \\
+ \frac{1}{s^2(s+\lambda_1)(s+\lambda_2)(s+\lambda_3)} b_{2} b_{3} b_{4} \lambda_1 \lambda_2 \lambda_3 \omega_{1} \\
+ \frac{1}{s^2(s+\lambda_2)(s+\lambda_3)} b_{3} b_{4} \lambda_2 \lambda_3 \omega_{2} \\
+ \frac{1}{s^2(s+\lambda_3)} b_{4} \lambda_3 \omega_{3} \\
+ \frac{1}{s^2} \omega_{4}
\end{split}
\end{equation}

By using partial fraction expansion and standard Laplace Transforms, the set of equations below is used to calculate the amount of the m\textsuperscript{th} isotope in the decay chain, providing the m\textsuperscript{th} isotope is unstable.

\begin{equation}
\begin{split}
N_{m}(t; \vec{\lambda}, \vec{b}, \vec{w})
= \sum_{k=1,m} r(k; \vec{\lambda}, \vec{b}) \left[ f(t; k,m,\vec{\lambda}) N_{k}(0) + g(t;k,m,\vec{\lambda}) w_{k} \right ]
\end{split}
\end{equation}

\begin{equation}
\begin{split}
r(k,m,\vec{\lambda}) =
\begin{cases}
\prod_{i=k,m-1} \left( b_{i+1} \lambda_{i} \right) , & \text{if } k < m\\
1, & \text{if }k = m
\end{cases}
\end{split}
\end{equation}

\begin{equation}
\begin{split}
f(t;k,m,\vec{\lambda})
=
(-1)^{m-k}
\sum_{i=k,m}
\left[
\exp(-\lambda_i t)
\prod_{j=k,m;j\neq i}
\left(
\frac{1}{\lambda_i-\lambda_j}
\right )
\right ]
\end{split}
\end{equation}

\begin{equation}
\begin{split}
g(t;k,m,\vec{\lambda})
= \frac{1}{\prod_{i=k,m} \lambda_i }
+ \left( -1 \right)^{m-k+1}
\sum_{i=k,m}
\left[
\frac{1}{\lambda_i }
\exp(-\lambda_i t)
\prod_{j=k,m;j\neq i}
\left(
\frac{1}{\lambda_i - \lambda_j}
\right )
\right]
\end{split}
\end{equation}

The set of equations below is used to calculate the amount of the m\textsuperscript{th} isotope in the decay chain, where the m\textsuperscript{th} isotope is stable.

\begin{equation}
\begin{split}
N_{m}(t; \vec{\lambda}, \vec{b}, \vec{w})
= N_{m} + w_{m} t +
\sum_{k=1,m-1} r(k; \vec{\lambda}, \vec{b}) \left[ f(t; k,m-1,\vec{\lambda}) N_{k}(0) + g(t;k,m,\vec{\lambda}) w_{k} \right ]
\end{split}
\end{equation}

\begin{equation}
\begin{split}
r(k,m,\vec{\lambda}) =
\begin{cases}
\prod_{i=k,m-1} \left( b_{i+1} \lambda_{i} \right) , & \text{if } k < m\\
1, & \text{if }k = m
\end{cases}
\end{split}
\end{equation}

\begin{equation}
\begin{split}
f(t;k,m,\vec{\lambda})
= \frac{1}{\prod_{i=k,m} \lambda_i }
+ \left( -1 \right)^{m-k+1}
\sum_{i=k,m}
\left[
\frac{1}{\lambda_i }
\exp(-\lambda_i t)
\prod_{j=k,m;j\neq i}
\left(
\frac{1}{\lambda_i - \lambda_j}
\right )
\right]
\end{split}
\end{equation}

\begin{equation}
\begin{split}
g(t;k,m,\vec{\lambda})
= \frac{1}{\prod_{i=k,m} \lambda_i } t
+ \frac{\sum_{i=k,m} \left[ \prod_{j=k,m; j \neq i} \lambda_{j} \right]}
{\prod_{i=k,m} \lambda_{i}^2}
+ \left( -1 \right)^{m-k+1}
\sum_{i=k,m}
\left[
\frac{1}{\lambda_i^2}
\exp(-\lambda_i t)
\prod_{j=k,m;j\neq i}
\left(
\frac{1}{\lambda_i - \lambda_j}
\right)
\right]
\end{split}
\end{equation}

\subsection{Preference: Analytic over Numeric}

\begin{figure}
	\begin{center}
		\includegraphics[width=15.0cm]{chapters/methodology_activity/plots/po-218/po218_po218.eps}
		\captionsetup{font={it}}
		\caption{Decay of Po-218: Analytice and Gaver-Stehfest Calculations \cite{jeff311}}
		\label{fig:po218decay}
	\end{center}
\end{figure}

The numeric solution only requires the equation to be solved in the s-domain; the Gaver-Stehfest algorithm performs the inversion.  It is worth the extra effort to derive and implement an analytic solution, as the numeric is only an approximation.  Examples of the pitfalls of the numeric solution are that it can give negative amounts of an isotope and the difference between the numeric and analytic calculated amounts can become quite large when the isotope decays away to a very small value.  Figure \ref{fig:po218decay} shows the predicted decay of a sample of Po-218 irradiated for 1,000s, and sampled until 10,000s.  In the region between 4,000s and 9,000s the amount from the numeric calculation drops below zero, whereas the analytic calculation remains above zero, as would be expected.



\section{Computational Methods}

The Activity program has been developed in Fortran and takes advantage of MPI (Message Parsing Interface) to speed up calculation times by allowing the use of multiple processes in parallel.  It has a self contained maths library, although this could be improved in the future by using optimised maths libraries for certain functions (e.g. matrix operations).

The code was developed on a Debian based distribution of Linux, but it should be supported on other variants of Linux and Unix, and does not require any specialist hardware.

\begin{figure}[!h]
	\centering
	\begin{tikzpicture}[node distance=2cm]
	% Row 1
	\node (1a) [process] {\makecell[l]{User input file\\read into memory}};
	\node (1b) [process, right of=1a, xshift=2.5cm] {\makecell[l]{Load cross section\\data for isotopes\\(specified by user)}};
	\node (1c) [process, right of=1b, xshift=2.5cm] {\makecell[l]{Load ion exyz\\trajectory file}};
	% Row 2
	\node (2a) [process, below of=1a] {\makecell[l]{Loop through time steps:\\calculate amount/activity\\of each isotope}};
	\node (2b) [process, right of=2a, xshift=2.5cm] {\makecell[l]{Split polynomial fit\\into segments: calculate\\all possible reaction rates}};
	\node (2c) [process, right of=2b, xshift=2.5cm] {\makecell[l]{Use MPI: Fit polynomial\\to trajectory of each ion\\and take superposition}};
	% Row 3
	\node (3a) [process, below of=2a] {\makecell[l]{Output data and\\charts to user}};
	
	% arrows
	\draw [thick,->] (1a) -- (1b);
	\draw [thick,->] (1b) -- (1c);
	\draw [thick,->] (1c) -- (2c);
	\draw [thick,->] (2c) -- (2b);
	\draw [thick,->] (2b) -- (2a);
	\draw [thick,->] (2a) -- (3a);
	%\draw [->] (isotope1) -- (isotope2);
	%\draw [->] (isotope2) -- (stable);
	\end{tikzpicture}
	\captionsetup{font={it}}
	\caption{Flow chart of major processes in the Activity code}
	\label{fig:processflowchart}
\end{figure}

The user is required to prepare an input file that contains the instructions required to perform a calculation.  In addition to the input file, the user must provide an EXYZ ion trajectory file output by SRIM.  Activity will read in the user input file, and the SRIM and data files listed within, before performing the calculation.  Figure \ref{fig:processflowchart} shows a flowchart of the major steps the code performs.

There are various settings in the user input file, but the main ones relating to the simulated experiment are:

\begin{itemize}
	\item Element composition of target (percentage by mass).
	\item Beam flux (current), energy, duration and area on target.
	\item Activity measurement time (end of the ``experiment").
	\item Material density.
	\item Target thickness.
\end{itemize}




\section{SRIM}

SRIM is an ion transport code.  








