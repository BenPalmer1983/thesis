\section{Proton Accelerators}

\subsection{Linac}

Since the development of the first linacs (linear accelerators) in the 1940s, their modern day versions have become some of the most powerful accelerators in the world.  The longest of linac, SLAC (), is 3.2km in length and it accelerates electrons and positrons at energies of up to 50GeV.  Several linacs for protons include the 800MeV linac component of the ISIS neutron source in Oxfordshire, and the 800MeV linac used by the Spallation Neutron Source at Oak Ridge National Laboratory.

The accelerator is constructed of several tubes, connected alternately to opposite terminals of a high frequency alternating current supply.  As protons enter the first tube, a negative voltage is applied.  As the protons reach the gap between the first and second tube, the polarity is reversed.  The positive charge that is now applied to the first tube pushes the protons forward as the negative charge on the second tube pulls the protons forward.  This process is repeated along the length of the accelerator, with the sections increasing in length due to the increase in velocity of the protons.

\subsection{Cyclotron}

Cyclotrons are reasonably compact and cost effective.  The largest current cyclotron, TRIUMF, is located in Canada and is able to output protons with energies over 500MeV.  It is relatively large, weighing 170 tons.  The University of Birmingham cyclotron is more compact and the protons it accelerates are in the 8-40MeV range.


\begin{table}[h]
\begin{center}
\begin{tabular}{c c c c c c}
Projectile & Energy (MeV) & Maximum Current (mirco A) \\
\hline \\
proton & 8-40 & 60 \\
deuteron & 8-40 & 30 \\
$He^{2+}$ & 8-53 & 30 \\
\end{tabular}
\end{center}
\caption{University of Birmingham Cyclotron Ion Beams}
\end{table}


\subsection{Synchrotron}

Two of the most well known accelerators are Synchrotrons: the Large Hadron Collider at CERN, and the now retired Tevatron at Fermilab.  


\subsection{Radio Frequency Quadrupole}
