
\subsection{Introduction}

Classical molecular dynamics simulations are used to study small volumes of a material, with model sizes typically in the range of $10^5$ to $10^6$ atoms and time periods on the picosecond scale \cite{tungstenmd}.  Such simulations have been used to model grain boundaries, and within these models irradiation damage may be simulated by initiating atom cascades that result from neutron radiation.  Trautt and Mishin used molecular dynamics to study grain boundary migration in copper (12), and a more relevant example is the study by Shibuta et al and their model of grain boundary energy in bcc iron-chromium (13).

The temperature and pressure of the model can also be controlled, and this is important because there may be a temperature dependence on the rate of depletion of PGMs at the grain boundary, if there is depletion at all.

One key ingredient to any classical molecular dynamics simulation is the interatomic potential used to describe the forces between atoms in the model.  Initially, simple pair potentials that described the force between two atoms were used.