
\section{Experiment, Modelling and Theory}

\subsection{Introduction}


Experiment: direct answers from physical reality
Limited by technology of the time, also limits due to Heisenberg uncertainty principle
Theory: state of the art theories have been replaced numerous times in the past
Quantum mechanics is a very accurate theory under certain circumstances
Some problems just too hard to solve with these theories
Modelling: bridges the gap
Experiment and theory have their flaws, so does modelling
Helps show what experiment can’t, and where theory is too hard to solve
Aim: take very accurate DFT calculations based on quantum theory, extrapolate to a larger scale by fitting EAM potentials, open a path to simulations of Pd + Fe and Ru + Fe

Design a size and time scale diagram





DIRAC 



Ab Intio:  PWscf, Vasp, Siesta 
hundreds to thousands of atoms

Molecular Dynamics: DLPOLY, LAMMP










\subsection{Simulating Materials on a Variety of Scales in Time and Space}

Stainless steel grain size less than 100 microns.  A 1 micron grain would contain tens of billions of atoms.  Given this number of atoms for a very small grain, it is very difficult to simulate just one grain over a short time period.

Certain properties derived ab initio assume the entire material is a single crystal, rather than made from grains of crystals.  





