Functional Forms of Embedded Atom Method Potentials

The functions used to represent the pair potential, electron density and embedding functional can be tabulated or calculated from analytic functions.  By using an analytic function, there is the advantage of being able to produce tabulated versions of the potential functions and functional if required.

In a derivation of an EAM potential for Iron by Mendelev et al (17) the authors use a hybrid function between two exponential functions and a polynomial spline to represent the pair potential, and polynomial splines to represent the electron density function and embedding functional.  Similarly, a potential has been derived for Uranium by Smirnova et al (18) that uses polynomial splines exclusively for the pair and density functions, and embedding functional.

Polynomial splines are attractive because they are continuous, have a continuous first order derivative for third order polynomials, a continuous second order derivative for fifth order polynomials, and they are flexible enough to create a potential that reproduces the forces predicted by Ab Initio.  Although each segment of the spline is determined by four parameters, the nodes between splines can be adjusted which reduces the number of parameters that are adjusted during the derivation process.

Two forms of polynomial spline have been used in the literature:

\begin{equation}
f(r) = \sum \limits_{i=1}^{n} A_{i}(r_{i}-r)H(r_{i}-r)
\end{equation}
\begin{equation}
f(r) = \sum \limits_{i=1}^{n} (a_{i}r^{3}+b_{i}r^{2}+c_{i}r+d_{i}) H(r_{i}^{upper}-r)H(r-r_{i}^{lower})
\end{equation}

The first type of spline has fewer parameters, but is a superposition of two or more polynomials up until the spline segment between the final two cutoff points.  This makes it tricky to fit the parameters to give a desired starting potential.  In the second type of spline, a single polynomial stitches between any two nodes.  The boundary conditions at each node are:
•	the output of the functions either side of the node be equal
•	the first and second derivatives of the polynomials either side of the node are equal at the node
While there are more parameters to fit for the second type of spline, the number can be reduced by varying the position of the nodes and calculating the parameters by fitting the polynomials between the nodes.
Given that we are deriving potentials for a metal alloy, a many body potential is needed.  The EAM potential is a more general form of the Finnis-Sinclair type potential, and it has been used in many molecular dynamics investigations into metals.  For these reasons, our aim is to derive EAM potentials for Fe-Pd and Fe-Ru.
Two-band method EAMs have additional degrees of freedom and may be useful as they can capture any changes in the potential as the species mix and form the alloy.  This is a secondary aim, the primary being standard EAM potentials.
Molecular dynamics codes are capable of using tabulated functions.  The form of the potential derived will be a spline of many polynomials that will be used to produce the tabulated versions of the potential functions.  These are more flexible than other analytical potentials, they can be easily splined to other functions (such as the hard core Ziegler Biersack Littmark [ZBL] hard core potential) and continuous first and second derivatives can be forced.
Four a fourth order polynomial spline, the functions all take the following form:

\begin{equation}
f(r) = \sum \limits_{i=1}^{n} (a_{i}r^{3}+b_{i}r^{2}+c_{i}r+d_{i}) H(r_{i}^{upper}-r)H(r-r_{i}^{lower})
\end{equation}


