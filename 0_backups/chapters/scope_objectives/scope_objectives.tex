\chapter{Thesis Objectives}

\begin{changemargin}{1.0cm}{1.0cm} 
\abstractpreamble{Chapter Summary}
\end{changemargin}
 
\section{Thesis Objectives}

This work is split into two parts.  The first is focused on the activation of targets irradiated by a proton beam.  The second derives an interatomic potential for Palladium and Iron.

\subsection{Thesis Objective & Results: Ion Irradiation Activity}

The Bateman equations allow the prediction of the amount of isotopes in a decay chain at a certain time, providing the starting amounts and decay constants for each isotope are known.  A new set of equations were derived to take into account multiple branching factors and continuous production rates of isotopes.  

A Fortran program called Activity was developed to calculated the radioactivity of ion irradiated samples, during and after irradiation.  A python module was later created to implement the new equations in a more user friendly way.

\begin{itemize}
\item https://github.com/BenPalmer1983/activity
\item https://github.com/BenPalmer1983/isotopes
\item https://github.com/BenPalmer1983/neutron\_activation
\end{itemize}



\subsection{Thesis Objective & Results: Ion Damage Simulations}

Two python programs were created to automate the process of converging the cut-off values for the planewave cut off, k-points and smearing for PWscf DTF calculations and to  calculate the equation of state, elastic constants and other properties of a crystaline material.  A third program, written in Fortran and Python, was developed to fit two-band embedded atom method potentials to DFT and experimental data using a genetic optimisation algorithm.

\begin{itemize}
\item https://github.com/BenPalmer1983/qeconverge
\item https://github.com/BenPalmer1983/qeeos
\item https://github.com/BenPalmer1983/eampa\_v3
\end{itemize}

As a result of the qeconverge and qeeos programs, the equation of state and bulk properties of pure FCC iron, a phase that does not exist at normal room temperatures and pressures, are calculated.  

The eampa program is used to fit interatomic potentials to the known and DFT calculated data for Iron and Palladium.

Working with Dr. Todorov at Daresbury Laboratory, the molecular dynamics code DL_POLY was modified to include the two-band embedded atom method in addition to the already present EAM functionality.


\subsection{Targets Not Achieved}

Using the derived potentials in either a molecular dynamics program or a kinetic monte carlo program, to study the effect of ion damage on an alloy of Iron and Palladium, would have been desired.  It soon became clear that the DFT calculations needed to fit the Iron-Palladium potential required higher numbers of k-point that initially expected.  This, combined with the fact that collinear spin calculations were selected due to the ferromagnetic and antiferromagnetic alignments of BCC and FCC iron atoms, lead to the majority of the time being used up to run the DFT calculations.  Unfortunately, this meant there was no time left to investigate ion damage through classical MD or kinetic monte carlo simulations.







