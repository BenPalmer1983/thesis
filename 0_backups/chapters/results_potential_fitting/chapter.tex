\chapter{Results: Interatomic Potential Fitting}

\begin{changemargin}{1.0cm}{1.0cm}
\abstractpreamble{Chapter Summary}
\end{changemargin}




\section{Fitting Only to Bulk Properties}

The fitting procedure is for both Iron and Palladium, but the FCC allotrope of Iron that is used in this work does not exists, at room temperature and pressure.  Experimental data was available for Palladium, but DFT generated data was required for Iron. 

\subsection{Palladium}

The input parameters were:

\begin{table}[ht]
\begin{tabular}{lccc}
\hline
Element & \multicolumn{3}{c}{PD}\\
Structure             & \multicolumn{3}{c}{Face Centered Cubic}\\
$a_0$                 & \multicolumn{3}{c}{3.89 Angstrom \cite{webelementspd}}\\
Nearest Neighbour     & \multicolumn{3}{c}{2.75 Angstrom \cite{webelementspd}}\\
Basis vectors         & 1.0 & 0.0 & 0.0 \\
                      & 0.0 & 1.0 & 0.0 \\
                      & 0.0 & 0.0 & 1.0         \\
$E_{coh}$             & \multicolumn{3}{c}{3.91 eV \cite{semiempiricalpots}}   \\
$B_0$                 & \multicolumn{3}{c}{195.5 GPA \cite{semiempiricalpots}}   \\
Elastic Constants     & $\begin{bmatrix} 243.4 & 145.0 & 145.0 & 0 & 0 & 0 \\ 145.0 & 243.4 & 145.0 & 0 & 0 & 0 \\ 145.0 & 145.0 & 243.4 & 0 & 0 & 0 \\ 0 & 0 & 0 & 116 & 0 & 0 \\ 0 & 0 & 0 & 0 & 116 & 0 \\ 0 & 0 & 0 & 0 & 0 & 116 \end{bmatrix}$ \\
\hline
\end{tabular}
\label{tab:multicol}
\caption{}
\end{table}

An analytic fit was chosen:


The final parameters were:





\subsection{Iron}


The input parameters were:

\begin{table}[ht]
\begin{tabular}{lccc}
\hline
Element & \multicolumn{3}{c}{PD}\\
Structure             & \multicolumn{3}{c}{Face Centered Cubic}\\
$a_0$                 & \multicolumn{3}{c}{3.89 Angstrom \cite{webelementspd}}\\
Nearest Neighbour     & \multicolumn{3}{c}{2.75 Angstrom \cite{webelementspd}}\\
Basis vectors         & 1.0 & 0.0 & 0.0 \\
                      & 0.0 & 1.0 & 0.0 \\
                      & 0.0 & 0.0 & 1.0         \\
$E_{coh}$             & \multicolumn{3}{c}{3.91 eV \cite{semiempiricalpots}}   \\
$B_0$                 & \multicolumn{3}{c}{195.5 GPA \cite{semiempiricalpots}}   \\
Elastic Constants     & $\begin{bmatrix} 243.4 & 145.0 & 145.0 & 0 & 0 & 0 \\ 145.0 & 243.4 & 145.0 & 0 & 0 & 0 \\ 145.0 & 145.0 & 243.4 & 0 & 0 & 0 \\ 0 & 0 & 0 & 116 & 0 & 0 \\ 0 & 0 & 0 & 0 & 116 & 0 \\ 0 & 0 & 0 & 0 & 0 & 116 \end{bmatrix}$ \\
\hline
\end{tabular}
\label{tab:multicol}
\caption{}
\end{table}

An analytic fit was chosen:


The final parameters were:






















As the density of austenitic stainless steel is similar to pure BCC Iron, the lattice parameter for FCC Iron was estimated based on the same density but a different structure.  This gave a starting value of 

semiempiricalpots





The Palladium crystal is FCC and cubic, giving




\section{EAMPA Fitting Code}

\section{Computer Package Development}
 
Fitting a potential requires many repeated calculations of the forces and total energy of configurations of hundreds of atoms using the potential as it is varied.  This puts a high demand on memory and central processing unit of a computer.  Python is an easy to use high-level programming language that supports object orientated programming.  Unfortunately, parallel programming using threading is hampered in Python by the global interpreter lock, and by it's very nature as an interpreted language, it is much slower than a compiled language.

There are several tools available to unlock the full potential of a modern multicore processor while writing in Python.

\subsection{Cython}



\subsection{F2PY}




\subsection{OpenMP}



\subsection{Development}

The resulting was developed using python and a shared object library written in Fortran 90 and compiled using F2PY.














