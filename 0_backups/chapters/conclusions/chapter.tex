\chapter{Conclusions}

\begin{changemargin}{1.0cm}{1.0cm}
\abstractpreamble{It is known that Chromim provides protection to stainless steel through a passive protective layer, but this becomes ineffective to protect the steel under irradiation, as chromium is depleted from the grain boundary.  Platinum Group Metals also provide a mechanism to protect steel from corrosion, and experimentaion or simulations may be used to investigate this.  If either of the two chosen PGMs, Palladium or Ruthenium, are depleted at the grain boundary under irradiation as well as depletion of Chromium, the steel will become susceptible to corrosion.  

Experimental testing is cheaper and easy if an ion beam is used to replicate neutron damage.  Either source will cause the target material to become radioactive.  It is important to be able to calculate how radioactive, and how long after irradiation (to the desired damage, in DPA) it will be until the material is safe to be handled.

Computer simulations to simulate radiation damage will be much too large and complex for first-principles calculations.  They will need either classical molecular dynamics or kinetic monte carlo simulations to encapsulate the much larger simulation volumes, to contain the atoms effected by the damage cascade, and to represent the grain boundary.  This requires derivation of an appropriate interatomic potential using experimental data and first-principles data where it is not available, or is either impossible or impractical to measure.}
\end{changemargin}






\section{Inter Granular Stress Corrosion Cracking}

Austenitic stainless steel is particularly succesptible to inter granular stress corrosion cracking.  It has good properties, including a good resistance to corrosion due to its high chromium and nickel content.  Under irradiation, the chromium at grain boundaries is depleted with the formation of Chromium carbides at the boundaries.  Three requirements for intergranular stress corrosion cracking to occur are:

\begin{itemize}
\item a susceptible material (austenitic stainless steel)
\item stress, welds, swelling, pressure, any applied stress (high pressure in reactor environment)
\item a corrosive environment (radiolysis of water)
\end{itemize}

The passive chromium layer, under normal conditions, protects the steel from corrosion.  The steel is made up from small grains, and the surface of these grains of crystalline metal are where the protective layer covers.  During irradiation, chromium is depleted, forming chromium carbides between the grain boundaries.  As the percentage of chromium drops, the surface becomes prone to corrosion.

Small quantities of platinum group metals may be added to a steel to also increase the corrosion resistance of the steel.  Platinum group metals are rare and much more costly that Iron, Nickel, Chromium.  Where irrdiation is not present, it has been shown experimentally that the benefit due to Palladium is lost due to the formation of PdMn nanoparticles, where Manganese is present in the steel, although no such particles are present when Palladium was replaced by Ruthenium.  It is expected that Palladium will not form such nano particles in steel without Manganese, but what about under irradiation?  Are the grain boundaries enriched with Platinum group metals, depleted or remain unchanged?


\section{Activity Code}

Testing a material inside a nuclear reactor is an expensive experiment.  There are many more ion sources around the world than high flux neutron sources, and ion beams are easier to direct due to the charge of the ion versus the neutral charge of the neutron.  The ion energies required to create similar sized damage cascades in the material are high enough to also have a chance of transmuting the target nuclei.  These transmuted nuclei are most likely going to be radioactive.  To reach damage dose levels comparable to a component in a GenIV reactor by the end of it's life (up to 200 DPA), the target material will become dangerously radioactive. 

Bateman's equation may be used to calculate the amount of an isotope in a decay chain of several isotopes, and thus how radioactive an isotope would be after a certain time.  Major modifications were required to be made to the equation to also include branching factors and souce rates for isotopes generated by an ion beam.

\section{Modified Equation and Activity Code}

Whilst the equation was first tackled by solving the differential equations head on, it became obvious that using Laplace transforms would be the logical choice; this was also the route Bateman followed when deriving the original equations.  Once transformed, a numerical algorithm (Gaver-Stehfest) was programmed and used to solve, with some success.  It isn't an exact solution, and the errors incurred often resulted in negative amounts of an isotope, and this was unacceptable.  More progress was made towards solving the problem analytically, and by using partial fractions, obviating the need for a numerical method.

The modified equation as derivied is an original contribution to knowledge.

The Activity was created to automate the process of calculating the activity of an ion irradiated target.  The program:

\begin{itemize}
\item reads in simulation details (target composition, thickness, duration etc)
\item reads in ion cross section data from the TENDL database
\item processes SRIM exyz data points
\item calculates the amount of each isotope by the end of the simulation
\item calculates the activity of each isotope and the dose an average human would be exposed to
\end{itemize}

A number of simplifications are made to the model, but predicted the radioactivity of certain peaks reasonably well with a sample of proton irradiated pure iron.  Given more time, the integration of the ion paths and cross section data would be revised.  A range of alloys would also be irradiated by a proton beam and, once their activity is measured at several time intervals after irradiation, they would compared to the activity predicted by the Activity code.

\section{Interatomic Potential: Iron-Palladium}

It would be too big a task to develop an interatomic potential that includes Iron, Palladium, Chromiun, Nickel and any other elements that make up a typical sample of austenitic stainless steel; Iron and Palladium are the starting point, with the possibility of adding more elements to the potential in the future.

Palladium exists in the FCC state at normal conditions, but Iron is BCC.  The relatively high percentage of Nickel in the steel stabilises the austenitic FCC structure.  The bulk properties, such as the lattice parameter, bulk modulus, elastic constants and so forth, are experimentally known for Palladium.  As iron doesn't exist in the gamma phase at normal conditions, DFT calculations were used to find the bulk properties of gamma iron.

\section{Collinear Spin DFT}

With most models, there will be a trade off between simplicity, accuracy and the computational cost of the model.  The parameters that had the largest impact on how long a DFT calculation will take are the planewave energy cutoff and number of k-point used to integrate the Brillouin zone.  These should be reduced as much as possible, while keeping the energy and force results within a tolerance determined by the user.

A choice in the complexity of the calculation also had to be decided upon.

Pure Iron under normal conditions has a BCC structure and energetically favours a ferromagnetic configuration.  Chromium also has a BCC structure under normal conditions, but it's optimum structure magnetically is antiferromagnetic.  Gamma phase Iron (FCC) is also antiferromagnetic, and so it will be important to include magnetism in the DFT calculations.

The options in increasing complexity, and computational time, are:

\begin{itemize}
\item non-magnetic
\item collinear - suitable for high symmetry, ferromagnetic/antiferromagnetic
\item non-collinear - spin may not be aligned in the same direction
\end{itemize}

For the Iron and Palladium calculations, collinear calculations were used and the starting configurations of the atoms were set up in accordance with the known configurations i.e. BCC Iron was configured such that the spins were aligned in the same direction, and FCC were set up in an antiferromagnetic setting.

A Python code was developed to automatically converge the aformentioned parameters.  Whilst it worked well for the planewave cutoff parameters, the choice of smearing type, amount and number k-points felt more like an art than a science.  It was more useful to select the number of k-points with a reasonable smearing width, within the convergence threshold set, and then to use those settings to calculate properties for the material and compare to the known values.

During testing, a much larger number of k-points were needed for the predicted properties of Aluminium to replicate the known properties reasonably well.  Unfortunately, the number of k-points used for Iron and Palladium was constrained by the amount of time and the RAM available on the compute nodes.  Ideally, more k-points would have been used, but this was not possible.



\section{Bulk Properties of Orthorhombic FCC Iron}

As pure Iron does not exist in its gamma phase at room temperature, DFT calculations were used in place of experimental data.  As eluded to in the previous section, collinear spin DFT was selected.  Although a smaller number of k-points were used, the calculation lasted for approximately 30 days to complete.

Rather that calculate the input files manually, a program was created to automate this process (QE\_EOS).  It was also designed to cache input and output files for the DFT program, PWscf, such that it would load an ouput file from the cache if it had already successfully run, to reduce the calculation time.

The relaxed, energy minimised, shape of the antiferromagnetic FCC iron was not cubic, but orthorhombic.  The equation of state calculation is specifically for a cubic crystal, but the elastic constants were calculated for the orthorhombic crystal.  From the 9 calculated elastic constants the melting point, bulk modulus, shear modulus and Youg's modulus were calculated.  These data are an original contribution to science.



\section{Two-Band EAM Contribution to DL-Poly}

During the early stages of this work it was clear that the Embedded Atom Method was a good candidate for the type of potential to be derived as it was well suited to modelling metals.  Work by Prof. Ackland at the University of Edinburgh considered using a separate density function and embedding functional to represent the S-Band and D-Band of transition elements such as Caesium.  Work by P. Olsson and J. Wallenius also used this two band version to model Iron-Chromium.  

After meeting with Prof. Todorov at Daresbury Laboratory, the source code for DL-Poly was editted to include new keywords, additional arrays to store the two-band density and embedding data for the second band and modifications to the energy and force subroutines.  This was then released with version 4.05 of DL_POLY in July 2013.  






\section{Palladium-Iron Configurations as Fitting Data}

To be able to fit the iron-palladium potentials, the bulk properties of both pure FCC iron and FCC palladium were either calculated or taken from experimental values.  The potentials would be trained to fit this data.  As well as this, a number of configurations of iron-palladium were generated with their atoms slightly displaced from their lattice positions.  This would give a wider range of atom seperations and force values for the fitting program to use.

The force data were entirely from first-principles calculations, and ideally there would have been more configurations, and they would have been computed with a higher number of k-points.  In addition, a wider range of vancancies, interstitials and mixing concentrations would have been desireable.  The latter point is limited due to the configuration sizes being only 32 atoms.  The atomic percentage on Palladium or Ruthenium doped is half a percent which would be better represented with a one PGM atom in a 4x4x4 FCC 256 atom supercell.














