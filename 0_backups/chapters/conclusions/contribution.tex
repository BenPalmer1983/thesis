\section{Contribution}

\subsection{Introduction}

The work in this thesis is split into two distinct parts, connected by the broader investigation into the irradiation of stainless steels (doped with platinum group metals).  The contributions made are an extension of existing knowledge and the creation of several computer codes that are useful to the scientific community.

\subsection{Activity Equation and Code}

The original Bateman equations were derived by Harry Bateman in 1910, in response to Ernest Rutherford's work on radioactive decay.  These equations allow the calculation of the amount of each isotope in a decay chain, at time t.  It may also be modified to allow for branching factors.

When irradiating steel with the cyclotron, it became apparent that after a short period of irradiation the activity was a concern.  A model of decay was created where by there was an additional source term for each isotope in the decay chain, as new isotopes are created throughout the duration of the irradiation.  

The extended equation was derived using a similar approach to Bateman using the Laplace domain to solve the equations.  These equations also calculate the amount of an isotope in the decay chain at time t, but they also allow decay branching, an initial amount of each isotope at time t=0 and a continuous source term for each isotope.




