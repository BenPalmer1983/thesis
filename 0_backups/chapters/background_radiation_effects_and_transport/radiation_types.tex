\section{Radiation Types Relevant to This Work}


\subsection{Introduction}

There are three types of radiation that are useful to discuss in this work, and two of these are of particular interest: Neutrons, Ions and Gammas.




\subsection{Protons and Ions}

Charged particles interact with matter through the Coulomb interaction.  As a charged particle passes through matter, it may interact with both the nucleus and electrons of an atom.  A sufficiently energetic ion may lose kinetic energy to electrons by either raising the electrons to a higher energy levels in the atoms, or by removing electrons from atoms altogether, ionizing atoms.  

Ions may also lose kinetic energy to the nucleus of an atom through elastic scattering and, where the atom is in a crystal structure, through knocking atoms in the material out of their lattice positions.

Knock on atoms and electrons with enough kinetic energy that have been removed from atoms (delta rays), continue the irradiation of the material while they have the kinetic energy available to do so.


\subsubsection{Ion Activation}

A large proportion of the energy of a charged projectile is lost to the electrons of the target material.  There is a chance, depending on the energy and type of charged projectile, and the cross section of the target nucleus, that the charged particle will overcome the coulomb potential and be captured by the nucleus.

This may result in a stable nucleus or an unstable nucleus.  If it is unstable, there is a probability that it will decay releasing energy in the form of photons, nucleons or electrons.  The initial ion irradiation creates sources of futher irradiation within the target material.




\subsection{Neutrons}


Neutrons interact with matter differently to that of protons, ions and other atoms, as the Neutron has no overall charge.  Neutrons do have a magnetic moment and experience a weak interaction with electrons, but the dominant interaction is between Neutrons and the Nucleus.  There are different methods in which Neutrons interact and these are determined by the kinetic energy, velocity and wavelength of the Neutron.  

 
\begin{table}[h]
\begin{center}
\begin{tabular}{c c c c}
\hline
Name & Energy Range & Velocity/ms\textsuperscript{-1} & Wavelength Ang \\
\hline
Cold & 0-0.025 eV & 0.0 - $2.2 \times 10^{3}$ &  $> 1.8$  \\
Thermal & 0.025 eV & ~$2.2 \times 10^{3}$ &  1.8  \\
Epithermal & 0.025-0.4 eV & $2.2 \times 10^{3}$ - $8.8 \times 10^{3}$ &  0.5-1.8 \\
Cadmium & 0.4-0.6 eV & $8.8 \times 10^{3}$ - $1.1 \times 10^4$ & 0.4-0.5  \\
Epicadmium & 0.6-1.0 eV & $1.1 \times 10^4$ - $1.4 \times 10^4$ & 0.3-0.4\\
Slow & 1-10 eV & $1.4 \times 10^4$ - $4.4 \times 10^4$   & 0.09-0.3\\
Resonance & 10-300 eV & $4.4 \times 10^4$ - $2.4 \times 10^5$ & 0.02-0.09\\
Intermediate & 300 eV - 1 MeV &  $2.4 \times 10^5$ & $2.9 \times 10^{-4}$ - $0.02$\\
Fast & 1-20 MeV & $1.4 \times 10^7$ - $6.1 \times 10^7$ & $6.5 \times 10^{-5}$ - $2.9 \times 10^{-4}$ \\
Relativistic & $>20$ MeV & $>6.1 \times 10^{7}$ & $< 6.5 \times 10^{-5}$\\
\end{tabular}
\end{center}
\caption{Neutron Categories by Energy Range \cite{njcarron}}
\end{table}








\subsubsection{Neutron-Neutron}



\subsubsection{Neutron-Proton}


\subsubsection{Neutron-Electron}




\subsection{Electrons}


\subsubsection{Delta Electrons and Delta Rays}




\subsection{High Energy Photons}

The electromagnetic spectrum classifies photons based on their energy and/or source, but visible light, x-rays, gamma rays and so on are all the same elementary 'particle'.  During the early years of Quantum Mechanics, the relationship between the energy and wavelength of a photon was discovered: the Planck-Einstein relation.

\begin{equation}
\begin{split}
E = h f
\end{split}
\end{equation}

\subsubsection{Pair Production}

The energy of photons in the database used in this work ranges from 1keV up to almost 10MeV.  There are several ways high energy photons will interact with the atoms of a target material.  The rest mass of an electron is 511keV.  If the photon energy is greater than 1.02MeV, i.e. there is at least enough energy to create an electron and positron, there is a chance that the photon will create an electron-proton pair.  

\begin{equation}
\begin{split}
h f = (m_e + m_p) c^2 + T_e + T_p
\end{split}
\end{equation}

The creation conserves energy and mass, with excess energy carried away as the kintetic energy of the particle pair.  The charge before the creation is zero, as the photon is neutral, and the charge after is also zero, with the -1 of the electron and +1 of the positron cancelling out.  Angular momentum is also conserved; the photon is a spin 1 Boson and, as electrons and positrons are Leptons they have half integer spin, adding up to 1.  Finally, momentum is not conserved in a vacuum, and this is why pair production occurs in the coulomb field of a nucleus.  The nucleus carries away excess momentum, fulfilling this conservation law.

\begin{figure}[h]
\begin{center}
\begin{tikzpicture}
  \begin{feynman}
    \vertex (a) {\(\gamma\)};
    \vertex [right=of a] (b);
    \vertex [above right=of b] (c) {\(e^-\)};
    \vertex [below right=of b] (d);
    \vertex [below right=of d] (e) {\(e^+\)};
    \vertex [below =of d] (f);
    \vertex [below left=of f] (g) {\(Z\)};
    \vertex [below right=of f] (h) {\(Z\)};
    \diagram* {
      (a) -- [boson] (b),
      (b) -- [fermion] (c),
      (b) -- [fermion] (d),
      (d) -- [fermion] (e),
      (d) -- [boson] (f),
      (g) -- [fermion] (f),
      (f) -- [fermion] (h),
    };
  \end{feynman}
\end{tikzpicture}
\end{center}
\end{figure}


\subsubsection{Compton Scattering}

An incident photon, with enough energy, may interact with the electron of an atom, an it transfers enough kinetic energy to eject the electron from the atom.  A lower energy photon is also created that carries away the remainder of the energy, but also linear momentum, as both energy and momentum must be conserved.


\begin{figure}[h]
\begin{center}
\begin{tikzpicture}
  \begin{feynman}
    \vertex (a) {\(\gamma\)};
    \vertex [below right=of a] (b);
    \vertex [right=of b] (c);
    \vertex [below left=of b] (d) {\(e^{-}\)};
    \vertex [above right=of c] (e) {\(\gamma\)};
    \vertex [below right=of c] (f) {\(e^{-}\)};
    \diagram* {
      (a) -- [boson] (b),
      (b) -- [fermion] (c),
      (d) -- [fermion] (b),
      (c) -- [boson] (e),
      (c) -- [fermion] (f),
    };
  \end{feynman}
\end{tikzpicture}
\end{center}
\end{figure}


\subsubsection{Photoelectric Effect}


\begin{figure}[h]
\begin{center}
\begin{tikzpicture}
  \begin{feynman}
    \vertex (a) {\(\gamma\)};
    \vertex [below right=of a] (b);
    \vertex [right=of b] (c);
    \vertex [below left=of b] (d) {\(Z_{n}^{+},e_{n}^{-}\)};
    \vertex [above right=of c] (e) {\(e_{1}^{-}\)};
    \vertex [below right=of c] (f) {\(Z_{n}^{+},e_{n-1}^{-}\)};
    \diagram* {
      (a) -- [boson] (b),
      (b) -- [fermion] (c),
      (d) -- [fermion] (b),
      (c) -- [fermion] (e),
      (c) -- [fermion] (f),
    };
  \end{feynman}
\end{tikzpicture}
\end{center}
\end{figure}



\subsubsection{Coherent Scattering}

Lower energy photons  


\begin{figure}[h]
\begin{center}
\begin{tikzpicture}
  \begin{feynman}
    \vertex (a) {\(\gamma\)};
    \vertex [below right=of a] (b);
    \vertex [right=of b] (c);
    \vertex [below left=of b] (d) {\(Z^{+}\)};
    \vertex [above right=of c] (e) {\(\gamma\)};
    \vertex [below right=of c] (f) {\(Z^{+}\)};
    \diagram* {
      (a) -- [boson] (b),
      (b) -- [fermion] (c),
      (d) -- [fermion] (b),
      (c) -- [boson] (e),
      (c) -- [fermion] (f),
    };
  \end{feynman}
\end{tikzpicture}
\end{center}
\end{figure}




