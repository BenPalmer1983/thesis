\chapter{Important Code Projects and Contributions}


\section{Activity Source Code}

This code was designed to calculate how radioactive a target becomes after being irradiated by protons.  It requires transport data from SRIM to be supplied by the user, but cross section data has been prepared and packaged with the program.

The key section of the source code has already been highlighted in the previously attached manual and paper.  The full source code and data needed to run the Activity code is available to download from github:

https://github.com/BenPalmer1983/activity



\section{Neutron Activity Source Code}

This code was created to estimate the activation and cooling of neutron irradiated targets.  There is no transport code, and a fixed Maxwell-Boltsmann distributed spectrum of neutron energies.

https://github.com/BenPalmer1983/neutron_activation




\section{Contribution to the DLPOLY Source Code}

I originally downloaded the 4.03 version of DL\_POLY while conducting my literature review.  As a result of reading work on  



\begin{lstlisting}[style=sEmail,caption={Add two numbers function}]
NEW FEATURES & IMPROVEMENTS
---------------------------

1.  New two band (2B) EAM and EEAM potentials for metals (TEST45 and
    TEST46). 

Acknowledgements
----------------
Ben Palmer @ University of Birmingham (UK) for contributing to the
development and testing of the 2BEAM for metals; 

\end{lstlisting}




\begin{lstlisting}[style=sFortran,caption={Add two numbers function}]
Subroutine metal_table_read(l_top)

!!!!!!!!!!!!!!!!!!!!!!!!!!!!!!!!!!!!!!!!!!!!!!!!!!!!!!!!!!!!!!!!!!!!!!!!
!
! dl_poly_4 subroutine for reading potential energy and force arrays
! from TABEAM file (for metal EAM & EEAM forces only)
!
! copyright - daresbury laboratory
! author    - w.smith march 2006
! amended   - i.t.todorov march 2016
! contrib   - r.davidchak (eeam) june 2012
! contrib   - b.palmer (2band) may 2013
!
!!!!!!!!!!!!!!!!!!!!!!!!!!!!!!!!!!!!!!!!!!!!!!!!!!!!!!!!!!!!!!!!!!!!!!!!

  Use kinds_f90
  Use comms_module, Only : idnode,mxnode,gsum
  Use setup_module, Only : ntable,nrite,mxgmet,engunit
  Use site_module,  Only : ntpatm,unqatm
  Use metal_module, Only : ntpmet,tabmet,lstmet,vmet,dmet,dmes,fmet,fmes
  Use parse_module, Only : get_line,get_word,lower_case,word_2_real

  Implicit None

  Logical, Intent( In    ) :: l_top

  Logical                :: safe
  Character( Len = 200 ) :: record
  Character( Len = 40  ) :: word
  Character( Len = 4   ) :: keyword
  Character( Len = 8   ) :: atom1,atom2
  Integer                :: fail(1:2),i,j,ipot,numpot,ktype,ngrid, &
                            cp,cd,cds,ce,ces,katom1,katom2,keymet,k0,jtpatm
  Real( Kind = wp )      :: start,finish

  Integer,           Dimension( : ), Allocatable :: cpair, cdens,cdnss, &
                                                    cembed,cembds
  Real( Kind = wp ), Dimension( : ), Allocatable :: buffer

  fail=0
  If      (tabmet == 1) Then ! EAM
     Allocate (cpair(1:(ntpmet*(ntpmet+1))/2),cdens(1:ntpmet),                              &
                                              cembed(1:ntpmet),                  Stat=fail(1))
  Else If (tabmet == 2) Then ! EEAM
     Allocate (cpair(1:(ntpmet*(ntpmet+1))/2),cdens(1:ntpmet**2),                           &
                                              cembed(1:ntpmet),                  Stat=fail(1))
  Else If (tabmet == 3) Then ! 2BEAM
     Allocate (cpair(1:(ntpmet*(ntpmet+1))/2),cdens(1:ntpmet),cdnss(1:ntpmet*(ntpmet+1)/2), &
                                              cembed(1:ntpmet),cembds(1:ntpmet), Stat=fail(1))
  Else If (tabmet == 4) Then ! 2BEEAM
     Allocate (cpair(1:(ntpmet*(ntpmet+1))/2),cdens(1:ntpmet**2),cdnss(1:ntpmet**2), &
                                              cembed(1:ntpmet),cembds(1:ntpmet), Stat=fail(1))
  End If
  Allocate (buffer(1:mxgmet),                                                    Stat=fail(2))
  If (Any(fail > 0)) Then
     Write(nrite,'(/,1x,a,i0)') 'metal_table_read allocation failure, node: ', idnode
     Call error(0)
  End If
  cpair=0 ; cp=0
  cdens=0 ; cd=0
  cembed=0 ; ce=0
  If (tabmet == 3 .or. tabmet == 4) Then
    cdnss=0 ; cds=0
    cembds=0 ; ces=0
  End If

  If (idnode == 0) Open(Unit=ntable, File='TABEAM')

! skip header record

  Call get_line(safe,ntable,record)
  If (.not.safe) Go To 100

! read number of potential functions in file

  Call get_line(safe,ntable,record)
  If (.not.safe) Go To 100
  Call get_word(record,word)
  numpot = Nint(word_2_real(word))

  Do ipot=1,numpot

! read data type, atom labels, number of points, start and end

     Call get_line(safe,ntable,record)
     If (.not.safe) Go To 100

! identify data type

     Call get_word(record,keyword)
     Call lower_case(keyword)
     If      (keyword == 'pair') Then
          ktype = 1
     Else If (keyword == 'dens' .or. keyword == 'dden') Then
          ktype = 2
     Else If (keyword == 'embe' .or. keyword == 'demb') Then
          ktype = 3
     Else If (keyword == 'sden') Then
          ktype = 4
     Else If (keyword == 'semb') Then
          ktype = 5
     Else
          Call error(151)
     End If

! identify atom types

     Call get_word(record,atom1)
     If (ktype == 1 .or.                                        & ! pair
         (ktype == 2 .and. (tabmet == 2 .or. tabmet == 4)) .or. & ! den for EEAM and dden for 2BEEAM
         (ktype == 4 .and. (tabmet == 3 .or. tabmet == 4))) Then  ! sden for 2B(EAM and EEAM)
        Call get_word(record,atom2)
     Else
        atom2 = atom1
     End If

! data specifiers

     Call get_word(record,word)
     ngrid = Nint(word_2_real(word))
     Call get_word(record,word)
     start  = word_2_real(word)
     Call get_word(record,word)
     finish = word_2_real(word)

! check atom identities

     katom1=0
     katom2=0

     Do jtpatm=1,ntpatm
        If (atom1 == unqatm(jtpatm)) katom1=jtpatm
        If (atom2 == unqatm(jtpatm)) katom2=jtpatm
     End Do

     If (katom1 == 0 .or. katom2 == 0) Then
        If (idnode == 0 .and. l_top) &
           Write(nrite,'(a)') '****',atom1,'***',atom2,'**** entry in TABEAM'
        Call error(81)
     End If

! store working parameters

     buffer(1)=Real(ngrid+4,wp) ! as if there are 4 extra elements after finish
     buffer(4)=(finish-start)/Real(ngrid-1,wp)
     buffer(2)=start-5.0_wp*buffer(4)
     buffer(3)=finish

     If (idnode == 0 .and. l_top) &
        Write(nrite,"(1x,i10,4x,2a8,3x,2a4,2x,i6,1p,3e15.6)") &
        ipot,atom1,atom2,'EAM-',keyword,ngrid,start,finish,buffer(4)

! check array dimensions

     If (ngrid+4 > mxgmet) Then
        Call warning(270,Real(ngrid+4,wp),Real(mxgmet,wp),0.0_wp)
        Call error(48)
     End If

     keymet=(Max(katom1,katom2)*(Max(katom1,katom2)-1))/2 + Min(katom1,katom2)
     k0=lstmet(keymet)

! check for undefined potential

     If (k0 == 0) Call error(508)

! read in potential arrays

     Do i=1,(ngrid+3)/4
        j=Min(4,ngrid-(i-1)*4)
        If (idnode == 0) Then
           Read(Unit=ntable, Fmt=*, End=100) buffer(4*i+1:4*i+j)
        Else
           buffer(4*i+1:4*i+j)=0.0_wp
        End If
     End Do

     If (mxnode > 1) Call gsum(buffer(5:ngrid+4))

! copy data to internal arrays

     If       (ktype == 1) Then

! pair potential terms

! Set indices

!        k0=lstmet(keymet)

        cp=cp+1
        If (Any(cpair(1:cp-1) == k0)) Then
           Call error(509)
        Else
           cpair(cp)=k0
        End If

        vmet(1,k0,1)=buffer(1)
        vmet(2,k0,1)=buffer(2)
        vmet(3,k0,1)=buffer(3)
        vmet(4,k0,1)=buffer(4)

        Do i=5,mxgmet
           If (i-4 > ngrid) Then
             vmet(i,k0,1)=0.0_wp
           Else
             buffer(i)=buffer(i)*engunit
             vmet(i,k0,1)=buffer(i)
           End If
        End Do

! calculate derivative of pair potential function

        Call metal_table_derivatives(k0,buffer,Size(vmet,2),vmet)

! adapt derivatives for use in interpolation

        Do i=5,ngrid+4
           vmet(i,k0,2)=-(Real(i,wp)*buffer(4)+buffer(2))*vmet(i,k0,2)
        End Do

     Else If (ktype == 2) Then

! density function terms
! s-density density function terms for EAM & EEAM
! d-density density function terms for 2B(EAM & EEAM)

! Set indices

        If      (tabmet == 1 .or. tabmet == 3) Then ! EAM
           k0=katom1
        Else If (tabmet == 2 .or. tabmet == 4) Then ! EEAM
           k0=(katom1-1)*ntpatm+katom2
        End If

        cd=cd+1
        If (Any(cdens(1:cd-1) == k0)) Then
           Call error(510)
        Else
           cdens(cd)=k0
        End If

        dmet(1,k0,1)=buffer(1)
        dmet(2,k0,1)=buffer(2)
        dmet(3,k0,1)=buffer(3)
        dmet(4,k0,1)=buffer(4)

        Do i=5,mxgmet
           If (i-4 > ngrid) Then
             dmet(i,k0,1)=0.0_wp
           Else
             dmet(i,k0,1)=buffer(i)
           End If
        End Do

! calculate derivative of density function

        Call metal_table_derivatives(k0,buffer,Size(dmet,2),dmet)

! adapt derivatives for use in interpolation

        dmet(1,k0,2)=0.0_wp
        dmet(2,k0,2)=0.0_wp
        dmet(3,k0,2)=0.0_wp
        dmet(4,k0,2)=0.0_wp

        Do i=5,ngrid+4
           dmet(i,k0,2)=-(Real(i,wp)*buffer(4)+buffer(2))*dmet(i,k0,2)
        End Do

     Else If (ktype == 3) Then

! embedding function terms
! s-density embedding function terms for EAM & EEAM
! d-density embedding function terms for 2B(EAM & EEAM)

! Set indices

        k0=katom1

        ce=ce+1
        If (Any(cembed(1:ce-1) == k0)) Then
           Call error(511)
        Else
           cembed(ce)=k0
        End If

        fmet(1,k0,1)=buffer(1)
        fmet(2,k0,1)=buffer(2)
        fmet(3,k0,1)=buffer(3)
        fmet(4,k0,1)=buffer(4)

        Do i=5,mxgmet
           If (i-4 > ngrid) Then
             fmet(i,k0,1)=0.0_wp
           Else
             buffer(i)=buffer(i)*engunit
             fmet(i,k0,1)=buffer(i)
           End If
        End Do

! calculate derivative of embedding function

        Call metal_table_derivatives(k0,buffer,Size(fmet,2),fmet)

     Else If (ktype == 4) Then

! s-density function terms

! The 2BM formalism for alloys allows for a mixed s-band density: rho_{atom1,atom2} /= 0
! (and in general for the EEAM it may be non-symmetric: rho_{atom1,atom2} may be /= rho_{atom2,atom2})
! Some 2BM models rho_{atom1,atom1}=rho_{atom2,atom2}==0 with rho_{atom1,atom2} /= 0
! whereas others choose not to have mixed s-band densities.

! Set indices

        If (tabmet == 3) Then ! 2BMEAM
!           k0=lstmet(keymet)
        Else If (tabmet == 4) Then ! 2BMEEAM
           k0=(katom1-1)*ntpatm+katom2
        End If

        cds=cds+1
        If (Any(cdnss(1:cds-1) == k0)) Then
           Call error(510)
        Else
           cdnss(cds)=k0
        End If

        dmes(1,k0,1)=buffer(1)
        dmes(2,k0,1)=buffer(2)
        dmes(3,k0,1)=buffer(3)
        dmes(4,k0,1)=buffer(4)

        If (Nint(buffer(1)) > 5) Then

           Do i=5,mxgmet
              If (i-4 > ngrid) Then
                 dmes(i,k0,1)=0.0_wp
              Else
                 dmes(i,k0,1)=buffer(i)
              End If
           End Do
! calculate derivative of density function

           Call metal_table_derivatives(k0,buffer,Size(dmes,2),dmes)

! adapt derivatives for use in interpolation

           dmes(1,k0,2)=0.0_wp
           dmes(2,k0,2)=0.0_wp
           dmes(3,k0,2)=0.0_wp
           dmes(4,k0,2)=0.0_wp

           Do i=5,ngrid+4
              dmes(i,k0,2)=-(Real(i,wp)*buffer(4)+buffer(2))*dmes(i,k0,2)
           End Do

        End If

     Else If (ktype == 5) Then

! s-embedding function terms

! Set index

        k0=katom1

        ces=ces+1
        If (Any(cembds(1:ces-1) == k0)) Then
           Call error(511)
        Else
           cembds(ces)=k0
        End If

        fmes(1,k0,1)=buffer(1)
        fmes(2,k0,1)=buffer(2)
        fmes(3,k0,1)=buffer(3)
        fmes(4,k0,1)=buffer(4)

        Do i=5,mxgmet
           If (i-4 > ngrid) Then
             fmes(i,k0,1)=0.0_wp
           Else
             buffer(i)=buffer(i)*engunit
             fmes(i,k0,1)=buffer(i)
           End If
        End Do

! calculate derivative of embedding function

        Call metal_table_derivatives(k0,buffer,Size(fmes,2),fmes)

     End If

  End Do

  If (idnode == 0) Close(Unit=ntable)
  If (idnode == 0 .and. l_top) Write(nrite,'(/,1x,a)') 'potential tables read from TABEAM file'

  If      (tabmet == 1 .or. tabmet == 2) Then ! EAM & EEAM
     Deallocate (cpair,cdens,cembed,              Stat=fail(1))
  Else If (tabmet == 3 .or. tabmet == 4) Then ! 2B(EAM & EEAM)
     Deallocate (cpair,cdens,cdnss,cembed,cembds, Stat=fail(1))
  End If
  Deallocate (buffer,                             Stat=fail(2))
  If (Any(fail > 0)) Then
     Write(nrite,'(/,1x,a,i0)') 'metal_table_read deallocation failure, node: ', idnode
     Call error(0)
  End If

  Return

! end of file error exit

100 Continue

  If (idnode == 0) Close(Unit=ntable)
  Call error(24)

End Subroutine metal_table_read

\end{lstlisting}







